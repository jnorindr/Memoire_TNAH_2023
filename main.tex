\documentclass[a4paper,12pt,twoside]{book}
\usepackage[T1]{fontenc}
\usepackage{inputenc}
\usepackage{fontspec}
\usepackage{lmodern}
\usepackage[english,french]{babel}
\usepackage{xspace} % Pour la gestion des espaces après les commandes
\usepackage{csquotes} % Gestion des guillemets dans la biblio et citations
\usepackage[xetex]{graphicx} % Package pour gérer les images
\usepackage{caption}
\usepackage{subcaption} % Package pour les subfigures
\usepackage{float} % Gestion de la position des images
\usepackage{listings} % Blocs de code
\usepackage{xcolor} % Couleurs des blocs de code
\usepackage{pdfpages} % Insertion de PDF dans le fichier

% Couleurs des blocs de code
\definecolor{ferngreen}{rgb}{0.31, 0.47, 0.26}
\definecolor{codegray}{rgb}{0.44, 0.5, 0.56}
\definecolor{sinopia}{rgb}{0.8, 0.25, 0.04}
\definecolor{silver}{rgb}{0.95, 0.95, 0.95}
\definecolor{bondiblue}{rgb}{0.0, 0.58, 0.71}

%Style des blocs de code
\lstdefinestyle{code-memoire}{
	backgroundcolor=\color{silver},   
	commentstyle=\color{codegray},
	keywordstyle=\color{sinopia},
	numberstyle=\tiny\color{codegray},
	stringstyle=\color{ferngreen},
	basicstyle=\ttfamily\footnotesize,
	breakatwhitespace=false,         
	breaklines=true,                 
	captionpos=b,                    
	keepspaces=true,                 
	numbers=left,                    
	numbersep=5pt,                  
	showspaces=false,                
	showstringspaces=false,
	showtabs=false,                  
	tabsize=2
}

\lstset{style=code-memoire}

\usepackage{hyperref}
\hypersetup{%
	pdfauthor={Jade Norindr}
	pdfsubject={Mémoire TNAH — Le traitement des sources historiques par la vision artificielle}, 
	pdfkeywords={vision artificielle, apprentissage profond, API, histoire des sciences}
}

% Mise en page École des chartes
\usepackage[margin=2.5cm]{geometry} % marges
\usepackage{setspace}
\onehalfspacing % Interligne de 1.5
\setlength\parindent{1cm}

% Bibliographies
\usepackage{tocbibind}
\usepackage[backend=biber, sorting=nyt, style=enc, minbibnames=10, maxbibnames=10]{biblatex}
\addbibresource{bibliographie/humanum.bib}
\addbibresource{bibliographie/ml.bib}
\addbibresource{bibliographie/technique.bib}
\addbibresource{bibliographie/astronomie.bib}
\addbibresource{bibliographie/websites.bib}
\nocite{*}
\defbibnote{intro}{Cette bibliographie présente toutes les ressources utilisées, de tout type, citées ou non, par simple ordre alphabétique.}


\author{Jade Norindr – M2 TNAH — ENC}
\title{Le traitement des sources historiques par la vision artificielle. L'exemple des manuscrits d'astronomie de tradition ptoléméenne}

% Acronymes
\usepackage[automake, acronym, toc]{glossaries}
\makeglossaries

\setacronymstyle{short-long}
\newacronym{api}{API}{\emph{Application Programming Interface}}
\newacronym{bnf}{BnF}{Bibliothèque nationale de France}
\newacronym{cnn}{CNN}{Convolutional Neural Nets}
\newacronym{cpu}{CPU}{\textit{Central Processing Unit}}
\newacronym{csv}{CSV}{\emph{Comma-separated values}}
\newacronym{dips}{DIPS}{\emph{DISHAS Interactive Parameter Squeezer}}
\newacronym{dishas}{DISHAS}{\emph{Digital Information System for the History of Astral Sciences}}
\newacronym{dti}{DTI}{\emph{DISHAS Table Interface}}
\newacronym{eida}{EIDA}{\emph{Editing and analysing hIstorical astronomical Diagrams with Artificial intelligence}}
\newacronym{enc}{ENC}{École nationale des chartes}
\newacronym{enherit}{EnHerit}{\emph{Enhancing Heritage Image Databases}}
\newacronym{gpu}{GPU}{\textit{Graphics Processing Unit}}
\newacronym{html}{HTML}{\emph{HyperText Markup Language}}
\newacronym{htr}{HTR}{\emph{Handwritten Text Recognition}}
\newacronym{http}{HTTP}{\emph{Hypertext Transfer Protocol}}
\newacronym{https}{HTTPS}{\emph{Hypertext Transfer Protocol Secure}}
\newacronym{iiif}{IIIF}{\emph{International Image Interoperability Framework}}
\newacronym{imagine}{IMAGINE}{Laboratoire d’Informatique Gaspard Monge}
\newacronym{inha}{INHA}{Institut national d'histoire de l'art}
\newacronym{iscd}{ISCD}{Institut des sciences du calcul et des données}
\newacronym{json}{JSON}{\emph{JavaScript Object Notation}}
\newacronym{ransac}{RANSAC}{\textit{RANdom SAmple Consensus}}
\newacronym{rest}{REST}{\textit{Representational state transfer}}
\newacronym{rmn}{Rmn-Grand Palais}{Réunion des musées nationaux-Grand Palais}
\newacronym{ssh}{SSH}{\textit{Secure Shell}}
\newacronym{svg}{SVG}{\emph{Scalable Vector Graphics}}
\newacronym{syrte}{SYRTE}{Systèmes de Référence Temps-Espace}
\newacronym{tei}{TEI}{\textit{Text Encoding Initiative}}
\newacronym{uri}{URI}{\emph{Uniform Resource Identifier}}
\newacronym{url}{URL}{\textit{Uniform Resource Locator}}
\newacronym{vhs}{VHS}{Vision artificielle et analyse Historique de la circulation de l'illustration Scientifique}
\newacronym{xml}{XML}{\emph{eXtensible Markup Language}}
\newacronym{yolo}{YOLO}{\textit{You Only Look Once}}

% Commandes
\newcommand{\api}{\gls{api}\xspace}
\newcommand{\bnf}{\gls{bnf}\xspace}
\newcommand{\cnn}{\gls{cnn}\xspace}
\newcommand{\cpu}{\gls{cpu}\xspace}
\newcommand{\cv}{\textit{computer vision}\xspace}
\newcommand{\dishas}{\gls{dishas}\xspace}
\newcommand{\dl}{\textit{deep learning}\xspace}
\newcommand{\docex}{docExtractor\xspace}
\newcommand{\eida}{\gls{eida}\xspace}
\newcommand{\enc}{\gls{enc}\xspace}
\newcommand{\enherit}{\gls{enherit}\xspace}
\newcommand{\exapi}{extractorAPI\xspace}
\newcommand{\gpu}{\gls{gpu}\xspace}
\newcommand{\http}{\gls{http}\xspace}
\newcommand{\https}{\gls{https}\xspace}
\newcommand{\ia}{intelligence artificielle\xspace}
\newcommand{\iiif}{\gls{iiif}\xspace}
\newcommand{\imagine}{\gls{imagine}\xspace}
\newcommand{\inha}{\gls{inha}\xspace}
\newcommand{\json}{\gls{json}\xspace}
\newcommand{\ml}{\textit{machine learning}\xspace}
\newcommand{\ponts}{École des Ponts ParisTech\xspace}
\newcommand{\rest}{\gls{rest}\xspace}
\newcommand{\ssh}{\gls{ssh}\xspace}
\newcommand{\svg}{\gls{svg}\xspace}
\newcommand{\uri}{\gls{uri}\xspace}
\newcommand{\URL}{\gls{url}\xspace}
\newcommand{\vhs}{\gls{vhs}\xspace}
\newcommand{\yolo}{\gls{yolo}\xspace}
\newcommand{\yolov}{YOLOv5\xspace}
\newcommand{\ist}{\textsc{i}\ieme{}\xspace}
\newcommand{\ii}{\textsc{ii}\ieme{}\xspace}
\newcommand{\viii}{\textsc{viii}\ieme{}\xspace}
\newcommand{\ix}{\textsc{ix}\ieme{}\xspace}
\newcommand{\xie}{\textsc{xi}\ieme{}\xspace}
\newcommand{\xii}{\textsc{xii}\ieme{}\xspace}
\newcommand{\xiii}{\textsc{xiii}\ieme{}\xspace}
\newcommand{\xv}{\textsc{xv}\ieme{}\xspace}
\newcommand{\xvi}{\textsc{xvi}\ieme{}\xspace}
\newcommand{\xviii}{\textsc{xviii}\ieme{}\xspace}
\newcommand{\jc}{av. J.-C.\xspace}
\newcommand{\ma}{Moyen-Âge\xspace}
\def\cdt{\kern-0.5pt\ensuremath\cdot\kern-0.5pt}

% Page à blanc sans en-tête et bas de page
\newcommand{\clearemptydoublepage}{\newpage{\pagestyle{empty}\cleardoublepage}}

% Pour des chapitres non numérotées dans la table des matière
\newcommand\chapterNo[1]{
  \chapter*{#1}
  \markright{\MakeUppercase{#1}}
}

\begin{document}
\onehalfspacing
\frontmatter

    \include{templates/page-titre}

    \thispagestyle{empty}	
    \cleardoublepage
	
    \chapterNo{Résumé}
\addcontentsline{toc}{chapter}{Résumé}
\medskip	

Résumé\\

\textbf{Mots-clés~:} histoire des sciences~; diagrammes astronomiques~; images~; \textit{machine learning}~; vision artificielle~; apprentissage profond~; YOLOv5~; réseau de neurones artificiels~; automatisation~; Python~; programmation modulaire~; IIIF~; API~; Flask.\\

\textbf{Informations bibliographiques~:} Jade Norindr, \textit{Le traitement des sources historiques par la vision artificielle. L'exemple des manuscrits d'astronomie de tradition ptoléméenne}, mémoire de master \og Technologies numériques appliquées à l'histoire~\fg, dir. Maxime Challon, École nationale des chartes, 2023.
	
\clearemptydoublepage
	
    \chapterNo{Remerciements}
    \addcontentsline{toc}{chapter}{Remerciements}
    
    Mes remerciements vont, en premier lieu, à toutes les personnes qui m'ont accompagnée lors de cette année de Master.
    
    Je tiens à adresser ma reconnaissance à mon directeur, Maxime Challon, pour son encadrement et pour ses conseils précieux tout au long de la rédaction de ce mémoire, qui ont permis son aboutissement. Je remercie chaleureusement Ségolène Albouy pour le rôle de tutrice qu'elle a joué lors de ce stage, pour sa bienveillance et son implication, qui m'ont permis de penser ce mémoire dans les meilleures conditions possible. Je remercie également Matthieu Husson pour la confiance qu'il m'a accordée, qui a rendu ce stage si enrichissant, et l'ensemble de l'équipe d'Histoire des sciences de l'Observatoire de Paris pour leur accueil.
    
    Merci ma famille et mes ami$\cdot$e$\cdot$s pour leur soutien lors de la rédaction de ce mémoire, à Ethan et Tara pour nos sessions de travail. Je remercie, enfin, Elsa et Arthur pour leurs relectures attentives.

    \printbibliography[keyword={DH},title={Humanités numériques}]
    \clearemptydoublepage
    \printbibliography[keyword={ml},title={\textit{Machine learning}, \textit{deep learning} et vision artificielle}]
    \clearemptydoublepage
    \printbibliography[keyword={technique},title={Documentation technique et méthode}]
    \clearemptydoublepage
    \printbibliography[keyword={astronomie},title={Histoire de l'astronomie}]
    \clearemptydoublepage
    
    \chapterNo{Introduction}
    \addcontentsline{toc}{chapter}{Introduction}

    \clearemptydoublepage

\mainmatter

    \part{Construire un corpus de numérisations pour le traitement par vision artificielle}
        \chapter{Le projet EiDA}
        Cette partie a pour objectif de revenir sur le contexte institutionnel du projet \eida, projet de recherche mené à l'Observatoire de Paris par l'équipe d'histoire des science du laboratoire \acrshort{syrte}. Ce projet a pour sujet d'étude les diagrammes astronomiques de tradition ptoléméenne : cette partie vise ainsi à contextualiser d'un point de vue historique le corpus du projet, et à en expliciter les bornes chronologiques et géographiques.
        
                \section{Contexte et objectifs du projet}
                    % Contexte et objectifs du projet

\subsection{Un projet de recherche en humanités numériques}
    \subsubsection{Cadre du projet}
	Les sciences astronomiques se développent, à travers les siècles, à l’intersection de besoins et de questionnements auxquels elles tentent de répondre. Des besoins politiques aux besoins religieux, ces disciplines existent au cœur de contextes culturels, géographiques, historiques variés et interconnectés, et produisent, dans le cadre de leur pratique, des supports éclectiques qui témoignent de l'évolution des idées, et des conceptions du monde et de l'univers. Les manuscrits et imprimés produits dans le cadre des sciences astronomiques, pour répondre aux besoins de ceux qui les emploient dans leurs pratiques scientifiques, astrologiques, scolaires, comportent des tables, des textes et des diagrammes d’une grande richesse, à l’image de la diversité de leurs contextes de production et de leurs utilisations. 
	
	Le projet \eida\footcite{EdIterAnalyserDiagrammes}, ayant pour vocation d'analyser ces sources dans le cadre de l'histoire des sciences, s’intéresse plus spécifiquement aux diagrammes astronomiques, en tant que patrimoine visuel et supports pour la circulation des idées et théories astronomiques et mathématiques. L’étude de ces circulations implique l’exploitation d’un corpus large, représentatif des bornes chronologiques et géographiques du projet, qui s’étendent de l’Europe à l’Asie, du \viii au \xviii siècle. Le projet \eida a pour objectif d’étudier la diversité des diagrammes produits, du point de vue de l’utilisation mais aussi du support et de la diffusion. Les diagrammes sont étudiés d’un point de vue épistémique, par la définition de différentes typologies de diagrammes, l’analyse de leurs utilisations dans divers contextes et l’étude des sujets mathématiques et astronomiques qui les implique ; ainsi que d’un point de vue documentaire, avec un intérêt pour leurs contextes de production, le rapport texte-diagramme dans les manuscrits, et la circulation des représentations à une échelle locale ou globale. 
	
	Le projet s’appuie sur des sources latines, byzantines, arabes, persanes, hébraïques, sanskrites et chinoises, qui partagent des éléments communs permettant ainsi de retracer les évolutions des sciences astronomiques, et d’étudier leur continuité, leur diversité, et leurs connections à travers l’histoire de l’astronomie.

    \subsubsection{Le choix de la vision artificielle}
	Profondément ancré dans les méthodes des humanités numériques, le projet \eida a pour objectif le développement d’une approche reposant sur la vision artificielle, qui permettrait d’appliquer au corpus une série de traitements en prévision de leur analyse et de leur exploitation. Le développement d’outils basés sur la vision artificielle a pour but de limiter les étapes manuelles de fouille et d’annotation dans le corpus de manuscrits, en automatisant les étapes d’extraction des diagrammes dans les numérisations, et leur transformation en objets aisément manipulables pour leur analyse, comme pour l’édition. Le projet \eida repose ainsi, dans ses principes fondateurs, sur une approche interdisciplinaire mêlant histoire des sciences et vision artificielle, avec pour objectif l’analyse de sources historiques dans le domaine des sciences astronomiques et le développement d’outils pour l’étude des diagrammes.
	
	Dans une perspective de science ouverte, le projet \eida prévoit le développement d’une application Web dédiée à l’extraction, la visualisation, l’étude et l’édition de diagrammes ; une plateforme dédiée à la communauté de la recherche qui permettra une utilisation des données produites et des outils développés par \eida via une interface web. Cette plateforme permettra à un public de chercheurs, d’étudiants ou d’amateurs d’exploiter les algorithmes développés dans le cadre du programme pour traiter leurs propres sources ; à l'image de la plateforme développée dans le cadre du projet \dishas, dédié aux tables astronomiques. 

    \subsubsection{Annotations et données ouvertes}
	Le projet \eida ayant pour objectif de produire un modèle de vision par ordinateur performant pour la détection de diagrammes dans des numérisations de manuscrits, un jeu de données d'entraînement -- ou vérité de terrain -- est produit à partir d'annotations réalisées par les chercheurs du programme. Ce jeu de données est composé d'un ensemble d'images et de leurs fichiers d'annotation où sont localisés manuellement les diagrammes, selon les résultats espérés de la détection automatique. Cette vérité de terrain se veut, pour des meilleurs résultats, la plus exhaustive possible quant au matériau source, et constitue ainsi un jeu de données diversifié en termes d'images et d'annotations, aussi bien du point de vue de la période que de la provenance géographique, ou simplement du style de représentation. 
	
	\eida prévoit ainsi de mettre à disposition ce jeu de données, pour permettre son exploitation par d'autres projets qui souhaiteraient entraîner un modèle de vision artificielle pour la détection d'objet dans des images. Il est envisagé, pour le bon référencement de ce jeu de données, de les référencer dans un catalogue à l'image de l'initiative HTR-United\footnote{HTR-United ne propose pas de cataloguer des jeux de données pour la \cv, il est cependant envisagé de prendre contact avec les porteurs de l'initiative pour discuter de la possible intégration des données du projet \eida. \cite{chagueHTRUnited}} pour la reconnaissance automatique des écritures manuscrites. L'annotation étant une étape cruciale des projets en \ml, l'ouverture de ces données permettra ainsi à d'autres projets de bénéficier du travail effectué dans le cadre d'\eida.

\subsection{Avant EiDA : le projet DISHAS}
    \subsubsection{Étudier les tables astronomiques}
	Précédant le projet \eida, \dishas s’intéresse au développement des sciences astronomiques, à leur circulation, à leur transmission à travers les cultures et les époques, par le prisme des tables. De la même manière que les diagrammes sont un objet d’étude permettant de retracer des traditions scientifiques à travers les cultures eurasiennes, les tables sont un objet caractéristique des sciences astronomiques, dont le contenu numérique rend possible l’alignement et la comparaison entre différentes sources pour une étude de la circulation des idées entre les cultures et les pratiques scientifiques.
	
	Le projet \dishas emploie les méthodologies des humanités numériques et les applique au traitement des sources en histoire de l’astronomie. Dans cette optique, une application dédiée à l’édition et à l’analyse des tables astronomiques a été mise en ligne\footcite{HomeDISHASProject} : cette application propose une contextualisation historique, chronologique et géographique des tables\footcite{HistoricalNavigationDISHAS}, ainsi qu’une série d’outils plus spécifiques, pour leur étude par un public divers, plus ou moins spécialisé. Parmi ces outils dédiés au traitement des tables, \acrfull{dti} est destiné à la saisie de tables numériques et \acrfull{dips} propose l’analyse mathématique de tables astronomiques anciennes ; mettant ainsi à disposition de la communauté de la recherche les résultats du projet, pour que les données comme les outils puissent être utilisés dans le cadre de recherches dans le domaine de l’histoire des sciences. 

    \subsubsection{Approche par la vision artificielle}
	L’approche des données visuelles par l’\ia est un élément clé du projet \dishas, ayant donné lieu au développement de tableTranscriber\footcite{dotTableTranscriberAutomaticPipeline2022}, un outil dédié à la détection par un algorithme de vision artificielle\footnote{Basé sur \cite{monnierDocExtractor2023}.} de la structure des tables, afin d’en transcrire le contenu manuscrit à l'aide d'un algorithme de \acrshort{htr}. L’objectif de tableTranscriber est ainsi d’automatiser la chaîne de traitement des numérisations de tables astronomiques manuscrites, afin d’obtenir des données au format \acrshort{xml}, \acrshort{csv} ou \acrshort{html}, aisément manipulables pour une analyse des sources. 
	
	Cette volonté collaborative dans le développement d’outils numériques est centrale au projet \eida qui, comme son prédécesseur \dishas, vise à produire une recherche communautaire, aux acteurs scientifiques divers, à l’intersection des humanités numériques et de l’histoire des sciences.

\subsection{Acteurs du projet}
    \subsubsection{SYRTE-PSL-Observatoire de Paris, CNRS}
	\eida est un projet interdisciplinaire mené en collaboration avec de nombreux acteurs et institutions, pour en faire un projet réellement collaboratif dans la réalisation de ses objectifs. Au sein de l'Observatoire de Paris, l'équipe est composée d'historiens et d'ingénieurs affiliés au \acrfull{syrte}, ayant pour la plupart collaboré au projet \dishas, et possédant donc une expérience de la recherche en histoire des sciences, et en humanités numériques. L'équipe compte également des chercheurs indépendants et collaborateurs scientifiques externes, qui assurent la diversité du corpus en contribuant à \eida par leur expertise dans des sources issues d'aires géographiques et chronologiques qui s'ajoutent à celles représentées dans l'équipe d'histoire des sciences de l'Observatoire de Paris.
	
	Du point de vue des institutions, le groupe de recherche en vision artificielle \imagine basé à l'\ponts, co-coordonne le projet \eida, et assure notamment les recherches en vision artificielle et apprentissage machine nécessaires à la réalisation des objectifs du projet. Le projet \vhs, coordonné par l'\acrlong{iscd}, est également partenaire d'\eida pour le développement des algorithmes de détection et de la plateforme dédiée aux chercheurs.

    \subsubsection{Algorithmes de vision par ordinateur : le groupe de recherche IMAGINE}
	L'équipe \imagine -- avec laquelle l'équipe d'histoire des sciences avait déjà collaboré pour le développement de tableTranscriber dans le projet \dishas -- apporte au projet \eida son expertise dans le domaine de l'apprentissage profond et de la vision artificielle appliqués aux humanités numériques. Les recherches menées entre 2018 et 2022 dans le cadre du projet \enherit\footcite{EnhancingHeritageImage}, dédié à l'application d'algorithmes de vision à l'histoire de l'art pour la reconnaissance de motifs dans les œuvres d'art, sont le socle des développements menés pour les projets \eida et \vhs. 
	
	Dans le cadre de leur participation à \eida, l'objectif des chercheurs d'\imagine est le développement de nouvelles approches\footnote{Basées sur les progrès récents de l'analyse par synthèse. \cite{monnierUnsupervisedLayeredImage2021}} pour permettre la décomposition et la vectorisation de dessins techniques. Ces nouvelles approches par vision par ordinateur devraient permettre l'analyse et l'édition de diagrammes astronomiques sans intervention humaine, avec une méthode plus flexible dans son apprentissage, qui allègerait ainsi les étapes d'annotation nécessaires à l'apprentissage profond.
	
	Ces recherches en apprentissage faiblement supervisé sont également menées dans le cadre du projet \vhs, auquel l'équipe \imagine contribue également, avec un accent mis sur la recherche de similarité dans les images et dans le texte, qui devraient permettre une navigation facilitée des corpus étudiés, et qui seront également exploitées dans \eida.

    \subsubsection{Le projet ANR VHS}
	Le projet \vhs réunit des chercheurs en histoire des sciences et en vision artificielle pour l'étude de la circulation des savoirs scientifiques au Moyen-Âge et à la période moderne. Coordonné par Alexandre Guilbaud, \vhs a pour objectif de développer des méthodes d'analyse de corpus illustrés par apprentissage profond, afin d'étudier la circulation, l'évolution et la réutilisation des images scientifiques dans le cadre de la transmission des savoirs à travers les époques et les zones géographiques, et d'analyser le rapport des images au contenu textuel des ouvrages dans lesquels elles figurent. Les méthodes développées par le projet ont pour objectif de limiter la nécessité des annotations pour l'obtention de résultats dans la recherche de similarité, qui permettraient ainsi de faciliter la navigation de corpus scientifiques par des regroupements et rapprochements basés sur les images et sur le texte.
	
	Ces recherches s'accompagnent du développement d'une application Web, qui mettra à disposition de la communauté scientifique les outils et données produites dans le cadre du projet \vhs, afin que ceux-ci puissent être exploités par d'autres équipes travaillant dans le domaine des études visuelles, ou souhaitant apporter une analyse sur des corpus illustrés par le prisme de la circulation des motifs et des idées. 
	
	 Les membres de l'équipe \imagine sont sollicités lors des prises de décision relatives aux travaux menées par les chercheurs\footnote{On considère l'annotation des diagrammes dans les manuscrits comme un travail de recherche.} et participent aux discussions relatives au développement de la plateforme. De plus, pour un développement plus fluide des applications \vhs et \eida, l'ingénieur chargé de ce développement pour le projet \vhs mène régulièrement ses travaux à l'Observatoire, pour faciliter les échanges entre les deux équipes. Ces collaborations permettent une mutualisation des recherches et expertises des différents projets, ainsi que le développement commun d'outils réutilisables qui bénéficient des regards variés et complémentaires des différents acteurs.

                \section{Sources primaires}
                    % Sources primaires

\subsection{L’astronomie ptoléméenne : naissance et diffusion}
    \subsubsection{\textit{L'Almageste} et le modèle ptoléméen}
	L'astronomie et les disciplines qui y sont liées sont, dès l'Antiquité, cultivées dans un grand nombre de cultures où se développent des pratiques variées, appuyées sur l'observation des objets célestes et le développement de systèmes mathématiques pour en prédire le comportement. Jusqu'à l'arrivée des théories de Nicolas Copernic au \xv siècle, les travaux de Ptolémée\footnote{Claude Ptolémée (v. 100-v. 170) est un astronome, mathématicien et géographe actif à Alexandrie au \ii siècle de notre ère. \textit{L'Almageste}, complété vers 150, comprend un quart de siècle d'observations des mouvements des objets célestes, expliqués par des systèmes mathématiques qu'il développe dans son œuvre. Le système ptoléméen désigne le modèle géocentrique de l'univers qu'il développe dans ses travaux. \cite{jonesPtolemya}.} influencent, à travers l'Eurasie, les productions des astronomes. L'œuvre de Ptolémée devient en effet dès le \ii siècle une référence en matière d'astronomie : développées sur les travaux de ses prédécesseurs grecs, les théories de Ptolémée approfondissent les productions de son temps dont il fait la synthèse dans ses travaux, et qu'il enrichit de ses propres observations. Si le modèle géocentrique est déjà établi avant l'achèvement de \textit{L'Almageste}, les travaux de Ptolémée perfectionnent la théorie des épicycles\footnote{Introduite par les astronomes de la Grèce antique, la théorie des épicycles permet, dans un modèle géocentrique, d'expliquer les changements de vitesse et de direction dans le mouvement observé des planètes, du Soleil et de la Lune. Selon cette théorie, les objets célestes se déplacent à vitesse uniforme sur un cercle appelé épicycle, dont le centre est lui-même en rotation sur un cercle centré sur la Terre appelé déférent. \cite{rousseauEpicyclesPtolemee}}, et parviennent à l'affiner avec une précision suffisante pour la prédiction des positions des planètes, notamment par l'introduction du point équant\footnote{En effet, la théorie des épicycles telle qu'elle existe avant l'intervention de Ptolémée ne permet pas de justifier précisément des mouvements observés des planètes. Ptolémée introduit ainsi la notion de point équant, point excentré par rapport au centre de la Terre, à partir duquel la vitesse de rotation des corps céleste est constante. Ce modèle géométrique permet de calculer avec précision le mouvement longitudinal des planètes. \cite{evansHistoryAstronomy}}.
	
	\textit{L'Almageste} est constitué d'observations et descriptions des procédures mathématiques appliquées par Ptolémée pour établir les paramètres de son modèle géométrique\footcite{evansHistoryAstronomy}. À ses explications textuelles s'ajoutent des tables de valeurs numériques calculées par Ptolémée qui permettent, en suivant ses explications, de calculer la position des planètes, en tant qu'application de ses théories. Dans les siècles qui suivent le développement du modèle ptoléméen, peu d'innovations ont lieu dans les sciences astronomiques de tradition hellénique, et les travaux produits prennent essentiellement la forme de commentaires des théories de Ptolémée, sans remise en cause de son modèle, qui reste la norme pendant plusieurs siècles et est diffusé à travers l'Eurasie par le biais de traductions.

    \subsubsection{Diffusion du modèle ptoléméen}
	Avant les Grecs, les Babyloniens pratiquent, dès le \ist millénaire \jc, une astronomie basée sur des calculs arithmétiques permettant de prédire la position des planètes. L'astronomie indienne hérite de ces théories des sciences astronomiques babyloniennes qu'elle mêle avec des méthodes locales, qui s'enrichissent plus tard des travaux de Ptolémée et de son prédécesseur Hipparque, dont les idées voyagent vers l'Est\footcite{evansHistoryAstronomy} par le biais de la Perse.
	
	À partir du \viii siècle, dans le monde islamique, se développe une pratique de l'astronomie à l'intersection des théories grecques, babyloniennes, perses et indiennes. Suivant le modèle des tables de Ptolémée, les astronomes arabes développent au \ix siècle des ouvrages de tables, appelés \textit{zīj}, dédiés au calcul des positions du Soleil, de la Lune et des planètes mêlant ces diverses traditions. La traduction en arabe -- pendant le califat abbasside\footnote{La dynastie sunite des Abbassides gouverne le monde musulman de 750 à 1258. La capitale du califat, Bagdad, est le siège d'entreprises de traduction vers l'arabe d'écrits scientifiques, notamment depuis le grec.} -- des écrits de Ptolémée permet aux astronomes du monde islamique d'enrichir leurs pratiques en l'inscrivant dans cette tradition, tout en améliorant certains paramètres calculés dans le cadre de son modèle pour en préciser les prédictions\footnote{L'écart temporel entre les calculs de Ptolémée et ceux des astronomes arabes du \ix siècle permettent également de mettre en évidence certains déplacements observables seulement sur un temps plus long, tels que la diminution de l'obliquité de l'écliptique.}. Si des critiques des travaux de Ptolémée émergent, notamment au \xie siècle dans les écrits d'Ibn al-Haytham, son modèle reste prédominant jusqu'aux travaux de Copernic\footcite{evansHistoryAstronomy}.
	
	Dans le monde latin, les théories grecques se sont peu transmises, et l'enseignement de l'astronomie s'appuie essentiellement sur les écrits de Pline l'Ancien. Au \xii et \xiii siècles, les textes grecs sont rendus disponibles par des initiatives de traduction : Gérard de Crémone, notamment, traduit \textit{L'Almageste}\footnote{Ainsi que d'autres versions arabes de textes grecs, tels que \textit{Du ciel} d'Aristote, ou les \textit{Éléments} d'Euclide.} de l'arabe vers le latin, participant activement à la redécouverte du modèle ptoléméen par les scientifiques du monde latin. La transmission des écrits et théories astronomiques ne se fait donc pas directement des Grecs aux Latins, mais a pour intermédiaires les traductions effectuées aux siècles prédécents dans le monde arabo-musulman. En parallèle, le développement des universités réintègre l'astronomie à l'enseignement des arts libéraux : dans ce contexte d'émulation, de nouveaux ouvrages sont publiés, et notamment des tables astronomiques, toujours basés sur le modèle ptoléméen.
	
	Les sources chinoises témoignent d'une pratique de l'astronomie aussi ancienne que le \ii millénaire \jc sous la forme de prédiction des éclipses du Soleil et de la Lune. Cette pratique est profondément liée à la culture impériale, et les éclipses -- puis plus tardivement, les planètes -- sont un outil d'analyse du règne d'un empereur. Plus proche des procédures babyloniennes que du modèle géométrique grec, l'astronomie chinoise s'appuie avant tout sur l'arithmétique pour ses prédictions, et s'intéresse principalement aux événements ponctuels -- tels que les comètes ou les éclipses -- plutôt qu'aux mouvements des corps célestes\footcite{evansHistoryAstronomy}.
	
	Jusqu'à l'émergence du modèle héliocentrique, et la mise au centre de la pratique des théories de Copernic, le modèle ptoléméen représente, dans de nombreux contextes culturels, le socle des méthodes des sciences astronomiques. Qu'il soit amélioré, contesté, ou enrichi de pratiques locales, il reste un standard de la conception de l'astronomie en Eurasie, et permet ainsi de retracer une tradition des pratiques à travers les siècles.

\subsection{Les diagrammes}
    \subsubsection{Une culture visuelle des sciences astronomiques}
	Dans le contexte eurasiatique, les supports détaillant les pratiques des astronomes circulent entre les siècles et les cultures, et sont enrichis ou adaptés aux pratiques autochtones de l'astronomie, en réponse aux besoins spécifiques de chaque contexte culturel. Les manuscrits et imprimés, supports de ces pratiques, sont les témoins de ces échanges intellectuels et permettent d'établir une histoire des idées et des méthodes héritées du modèle ptoléméen (fig. \ref{fig:modeles_lune}). Ces œuvres comprennent, en support des pratiques, des textes, des tables et des diagrammes, tous porteurs des méthodes et théories des sciences astronomiques : le projet \dishas ayant mené une étude sur les tables astronomiques, il semble naturel qu'\eida, dans la continuité de celui-ci, ait pour vocation de s'intéresser spécifiquement aux diagrammes. Les astronomes, dans leur pratique, n'hésitent pas à employer des modes d'expression visuels, au-delà du texte, qu'il convient d'étudier en parallèle de celui-ci pour en comprendre les interactions. Ainsi, il est nécessaire de développer des outils permettant d'appréhender les formes variées que peuvent prendre les sources en histoire de l'astronomie ; à l'image des récentes recherches en histoire des sciences, qui s'appliquent notamment à une étude des sources visuelles, non-discursives, de la pratique scientifique, et en considérant les illustrations comme des objets d'étude à part entière, au-delà d'un simple accompagnement du contenu textuel\footnote{\cite{jardineCriticalEditingEarlyModern2010}, p. 394}.
	
	En tant que vecteurs d'une culture visuelle de l'astronomie, les diagrammes sont, pour les astronomes, le support d'une pratique scientifique, et sont ainsi révélateurs de leurs méthodes, de leur contexte d'exercice et de leur conception de ces disciplines. Le projet vise à proposer une analyse de ces diagrammes qui souligne la variété de leurs fonctions et des traditions auxquelles ils appartiennent, tout en étudiant leurs modes de circulation à travers l'histoire de l'astronomie. Dans la diversité de leurs sphères de création, ces diagrammes reçoivent \og une légitimité en tant qu'instruments pédagogiques et vecteurs de la pensée\footnote{\og [...] legitimacy as instruments of pedagogy and as vehicle of thought \fg. \cite{hamburgerDiagramParadigmCrossCultural2022}, p. 1.}  \fg, et occupent ainsi dans le contexte d'une culture visuelle globale une place particulière.

	\begin{figure}[h]
		\begin{subfigure}{0.34\linewidth}
			\centering
			\includegraphics[width=5.5cm]{images/modele_lune_arabe.png}
		\end{subfigure}
		\hspace{1pt}
		\begin{subfigure}{0.30\linewidth}
			\centering
			\includegraphics[width=5cm]{images/modele_lune_byz.png}
		\end{subfigure}
  		\hspace{1pt}
  		\begin{subfigure}{0.30\linewidth}
			\centering
			\includegraphics[width=5cm]{images/modele_lune_latin.png}
		\end{subfigure}
		\caption{Diagrammes astronomiques de tradition arabe, byzantine et latine}
		\label{fig:modeles_lune}
	\end{figure}

	\subsubsection{Bornes géographiques et chronologiques}
	Ayant la volonté de produire une étude des circulations et évolutions des diagrammes astronomiques, \eida s'attache à l'étude de sources provenant de sphères géographiques et temporelles larges : les pratiques eurasiatiques des sciences astronomiques partagent des éléments communs qui justifient une étude au cadre vaste, pour proposer une analyse transculturelle des continuités et divergences dans l'histoire de l'astronomie comme dans la culture visuelle. Pour répondre à cette volonté, le projet \eida s'appuie sur cinq corpus aux provenances temporelles et géographiques diverses, provenant de traditions diverses également, dans une volonté de représentativité. 
	
	Les manuscrits astronomiques arabo-persans, produits du \viii au \xviii siècle, représentent à eux seuls un corpus d'une grande diversité, aux traditions diverses aussi bien du point de vue des pratiques que de la provenance géographique. Les travaux produits dans ces contextes circulent à travers l'Eurasie du \ma au début de la période moderne, et influencent de nombreuses pratiques. Les manuscrits d'astronomie latins médiévaux sont produits majoritairement entre le \xiii et le \xvi siècle dans un contexte universitaire, et sont influencés par les traductions faites de l'arabe au latin de textes de tradition arabe ou persane. Les manuscrits byzantins, produits entre les \ix et \xv siècles sont à l'intersection des traditions helléniques, latines et arabo-persanes, et influencent considérablement les manuscrits astronomiques latins du \xv et du \xvi siècle. Les manuscrits sanskrits, dont la diversité reflète les nombreuses traditions établies sur près de deux millénaires, témoignent à partir du \xie siècle d'influences helléniques puis arabo-persanes. Les sources chinoises, dont le support n'est pas nécessairement le manuscrit, prennent souvent la forme d'imprimés par blocs xylographiques dont les matrices sont réemployées à travers plusieurs témoins. Ces documents témoignent des influences arabo-persanes puis latines des pratiques astronomiques chinoises à partir de la dynastie Ming\footnote{La dynastie Ming règne sur la Chine de 1368 à 1644.}. Les sources étudiées dans le cadre d'\eida proviennent essentiellement des dynasties Ming et Qing\footnote{La dynastie Qing, dernière dynastie impériale chinoise, règne de 1644 à 1912.}.
	
	Les bornes de ces cinq corpus représentent les bornes chronologiques et géographiques du projet. S'étendant du \viii au \xviii siècle, les périodes et aides étudiées se veulent représentatives du contexte afro-eurasien de la circulation des idées et des images. Ces corpus représentent, dans chaque cas, plusieurs milliers de manuscrits. Plusieurs centaines de manuscrits sont numérisés pour chaque tradition représentée dans le projet, permettant ainsi la représentativité espérée dans le cadre du projet \eida. Ces numérisations, mises à disposition par les institutions patrimoniales qui conservent ces témoins, représentent les sources primaires du projet, sur lesquelles seront appliquées les traitements en prévision de l'analyse par les chercheurs.

		\\
		
		\eida est un projet aux bornes chronologiques, géographiques et thématiques vastes, pour permettre une étude sur un temps long et dans un contexte global de la circulation des diagrammes astronomiques et des théories scientifiques qui les accompagnent. Dans cette démarche, proposer une étude quantitative, traitant un grand nombre de sources, permet de mettre en avant des motifs, connexions et évolutions en accord avec la diffusion afro-eurasienne des idées développées par Ptolémée. Les bornes définies dans cette partie permettent de délimiter un corpus d'images -- numérisations d'ouvrages manuscrits ou imprimés -- qui en tant d'objets numériques présentent également leurs propres problématiques, sur lesquelles revient la partie suivante.
        \clearemptydoublepage
        
        \chapter{Images et interopérabilité}
        Les sources étudiées dans ce mémoire sont des sources iconographiques : cette partie vise à revenir sur les formats, ressources et méthodes pour le traitement des images en ligne, de la production de la ressource à sa publication. Les images digitales présentent des enjeux spécifiques, du point de vue de la technique, du droit et de la disponibilité. La mise en ligne et la diffusion d'une image fait suite à une longue chaîne de traitement qui a pour point de départ un objet matériel, et soulève des questionnements divers. Les spécifications \iiif tente de répondre à un certain nombre de problématiques liées à la présence en ligne de ressources iconographiques, et cette partie vise donc à présenter les solutions, possibilités et limites offertes par ce standard.
        
                \section{L’image comme source}
                    % L'image comme source

\subsection{Construire un corpus d’images : enjeux et méthodes}
    \subsubsection{La numérisation, enjeu patrimonial}
	L'application à un corpus de sources historiques de méthodes numériques repose sur une première étape cruciale dans le traitement de ces sources : la numérisation des collections patrimoniales. De la bibliothèque au musée, la numérisation du patrimoine culturel est un enjeu crucial depuis le début des années 1990\footcite{baujardNumerisationPatrimoineCulturel2017}. Pensée initialement comme un outil de préservation, de valorisation et d'accessibilité aux collections\footcite{richardProgrammeNumerisationBibliotheque1993}, la mise à disposition des collections au format numérique ouvre également la voie à de nouvelles méthodologies de la recherche, notamment dans le champs disciplinaire des humanités numériques : dans le cadre d'un projet tel qu'\eida, qui prévoit des traitements automatiques des sources par des algorithmes de vision par ordinateur, la disponibilité des sources numérisées est un principe fondateur de la démarche de recherche. 
	
	La numérisation des données culturelles fait désormais partie des pratiques courantes des institutions culturelles, et plus particulièrement des bibliothèques, représentant une part intégrante du travail de conservation et de diffusion du patrimoine\footcite{claerrModeEmploi2017}. Dès le milieu 1990, des initiatives telles que Gallica\footcite{Gallica}, bibliothèque numérique de la \bnf, voient le jour, et mettent à disposition du public, au format numérique, les collections de la bibliothèque. Dans les années qui suivent, des initiatives telles qu'Europeana\footcite{Europeana} permettent le déploiement de projets de plus large envergure\footcite{claerrModeEmploi2017}, portant toujours cette responsabilité de préservation, diffusion et valorisation des collections dans un contexte international. La notion de patrimoine culturel numérique prend tout son sens, et Internet devient un espace permettant à des utilisateurs aux profils variés d'exploiter ces ressources numériques, et d'accéder à un patrimoine vaste, au-delà des limites physiques de la consultation des documents. Cependant, malgré les initiatives de soutien à la numérisation portées par les collectivités, ou même par l'État\footcite{claerrModeEmploi2017}, il est nécessaire de souligner les disparités, aussi bien sur le plan national qu'international, de ces entreprises de numérisation, souvent coûteuses : ainsi, il n'existe pas de réelle égalité entre les institutions dans la numérisation et la mise en ligne des collections, notamment du point de vue des financements, de la qualité des données produites et du volume de documents traités. Il s'agit d'un biais qu'il est nécessaire de prendre en compte, et particulièrement dans des projets de recherche appuyés sur des sources aux bornes géographiques internationales. 
	
	Dans le cas des corpus étudiés par le projet \eida, on compte, pour des sources qui représentent plusieurs centaines de milliers de documents, une somme de plusieurs centaines de documents numérisés, mis à disposition par les institutions patrimoniales européennes, chinoises et indiennes : nous constatons ainsi que les documents numérisés ne représentent qu'un fragment de la réalité des sources existantes, cependant, ces centaines de documents numérisés sont à la fois suffisantes et suffisamment représentatives pour le bon déroulé du projet.
	
	Les collections patrimoniales des musées présentent un tout autre enjeu en termes de mise en ligne : les œuvres conservées, allant bien au-delà du format livre, ont une diversité qui redéfinit entièrement le processus de numérisation\footnote{Par numérisation, dans le contexte muséal, nous entendons essentiellement la prise de photographies de haute qualité des œuvres d'arts.}. Tout comme les bibliothèques, les institutions muséales font face, avec le développement des pratiques numériques, à une transformation des pratiques de conservation pour inclure ces méthodes. Outre les aspects documentaires de la numérisation, le Web est un lieu d'exposition à part entière, et la numérisation des œuvres présente ainsi un intérêt stratégique en termes de valorisation et de visibilité\footcite{baujardNumerisationPatrimoineCulturel2017}. Dans le monde muséal, néanmoins, l'accès libre aux images n'est pas aussi systématique qu'en bibliothèque\footnote{Pour comparaison, Gallica représente aujourd'hui près de 10 millions de documents, et la base de la \acrfull{rmn} compte 600 000 œuvres numérisées.} : il existe bel et bien des projets de numérisation de grande envergure, tels que Google Arts \& Culture\footcite{GoogleArtsCulture}, publié en 2011. La vocation de ce type de projets, cependant, semble plutôt être la création d'espaces d'exposition dématérialisés permettant de mettre en valeur les œuvres pour des spectateurs, plutôt que la véritable mise à disposition de ces images pour un usage par des utilisateurs. Nous soulignons, notamment, dans le cas de Google Arts \& Culture, l'impossibilité de copier les images avec un clic droit, et l'absence de bouton pour le téléchargement des numérisations sur un certain nombre de pages : les numérisations sont faites pour être vues, et non utilisées\footnote{Certaines œuvres du projet sont disponibles sur Wikimedia, mais ces images en accès libre ne représentent qu'un fragment des œuvres numérisées dans le cadre du projet. De plus, les musées français sont très peu représentés dans cette sélection. \cite{CategoryGoogleArt}}. 
	
	Une étude menée entre 2019 et 2020 par le Musée National de Tokyo souligne, en parallèle, les disparités dans la qualité des numérisations, qui ne permet ainsi pas de considérer qu'un objet numérisé ou photographié l'est nécessairement dans une qualité exploitable\footcite{sakaiDigitizingDisparityMuseum2021}, notamment parce que les besoins en termes de qualité ne sont pas les mêmes pour des œuvres en 2D et des œuvres en 3D, qui souffrent particulièrement de la mauvaise qualité des numérisations. Ainsi, en prenant en compte ces critères, il est nécessaire de soulever la question de la réelle exploitabilité des images mises en ligne, et particulièrement dans le cadre de projets en vision artificielle qui reposent, dans leurs fondations, sur la disponibilité et la qualité des images du corpus choisi. Des initiatives récentes tentent cependant de pallier à ces lacunes des institutions patrimoniales quant à la mise à dispostion des images, comme la création en 2015 d'une \api donnant accès aux reproductions de la base photographie de la \acrshort{rmn}\footcite{APIRMNGrandPalais}, mais la norme, pour ces images d'œuvres d'art, reste loin de l'\textit{open access}\footcite{mancaNouveauxDefisAgences2018}. 

    \subsubsection{Droit d'auteur et coût des images}
	La numérisation et la mise en ligne de documents patrimoniaux implique également de prendre en compte la question des droits d'auteur, qui régit souvent les possiblités des institutions en termes de libre accès, et impacte par conséquent les projets, aussi bien du point de vue des exploitations possibles que de la publication de leurs résultats\footcite{jacquotDecrireTranscrireDiffuser2017}. 
	
	Une œuvres originale est protégée au titre du droit d'auteur : ce droit comprend le droit moral, perpétuel et inaliénable, qui donne droit au respect de l'œuvre et du nom de son auteur, ainsi que le droit patrimonial, qui donne à l'auteur et ses ayants droit le droit d'autoriser ou non la reproduction et la diffusion de l'œuvre\footcite{sepetjanRespecterDroitPropriete2017}. Les droits patrimoniaux s'éteignent soixante-dix ans après la mort de l'auteur, et l'œuvre entre alors dans le domaine public\footnote{Il existe des cas particuliers, notamment pour les œuvres collectives ou les journaux, ainsi que pour les publications posthumes. \cite{GuidePratiquePour}.}. Il n'existe pas d'exception au droit d'auteur pour une numérisation mise en ligne sur Internet\footcite{sepetjanRespecterDroitPropriete2017}. Ces droits, qui s'appliquent à l'œuvre originale, ne sont pas nécessairement ceux qui nous intéressent dans le cadre d'un projet de recherche en histoire, dont les œuvres du corpus ont souvent rejoint le domaine public. La photographie d'une œuvre, cependant, peut faire l'objet du droit d'auteur, si celle-ci n'est pas considérée comme une reproduction servile\footnote{Pour que la photographie d'une œuvre d'art ne soit considérée une reproduction servile, il est nécessaire que celle-ci témoigne d'une intervention artistique de la part du photographe. Un choix de composition, de lumière ou un traitement avec un logiciel peuvent être considérés comme des interventions artistiques.}.
	
	Outre le droit d'auteur et les droits du photographe, les institutions peuvent exiger une redevance de réutilisation ou de prestation d'un service pour l'utilisation de leurs images, qui sert à couvrir, par exemple, les frais engagés pour la production du cliché, le traitement de la demande, ou la recherche iconographique\footcite{GuidePratiquePour}. Beaucoup d'institution, telles que la \bnf et l'\inha\footcite{denoyelleProposCoutImages2021}, proposent des tarifications particulières pour des projets académiques, scientifiques ou non commerciaux, et n'appliquent un coût qu'en cas de demande de numérisation d'œuvres qui ne sont pas déjà numérisées. Les projets de recherche peuvent ainsi faire le choix de construire leurs corpus en prenant en compte les documents déjà disponibles, et mis en ligne sous licence libre\footnote{Pour permettre aux chercheurs de traiter leurs numérisations sous droits avec les outils de la plateforme \eida, il a été décidé de proposer une option -- sous la forme d'un champs à cocher dans le formulaire de saisie -- pour que la numérisation envoyée ne soit pas rendue publique.}.

\subsection{Images et données ouvertes}
    \subsubsection{Licences Creative Commons}
	Pour favoriser la libre circulation des images sur le Web, des licences ont été créées pour encadrer la réutilisation des numérisations. Dès 2001, les licences Creative Commons\footnote{Ces licences ne sont plus reconnues par l'État français depuis 2016. \cite{denoyelleSavoirLicenceEtalab2021}.} sont créées avec la vocation de favoriser la diffusion et le partage des images dans un contexte numérique, en assouplissant notamment les droits de propriété intellectuelle pour s'adapter à ce nouveau cadre. Ces licences se basent sur un système de quatre options qu'il est possible de combiner pour créer six licences différentes. Ces options sont attribution (BY), non commercial (NC), sans modification (ND) et partage à l’identique (SD)\footcite{GuidePratiquePour}. 
	
	À partir de 2011, l'État français créée ses propres licences ouvertes, comme la licence unique Etalab, qui vise à \og permettre et encourager la réutilisation des données publiques\footcite{denoyelleSavoirLicenceEtalab2021} \fg. Une deuxième version, publiée en 2017, se veut compatible avec la licence CC BY, c'est-à-dire une licence ouverte avec attribution de l'œuvre à son auteur.
	
	Les données ouvertes sont un enjeu central des politiques culturelles : dans un contexte patrimonial, ces licences libres, conçues pour adapter les droits de propriété intellectuelle au contexte particulier du numérique, permettent ainsi la circulation, la diffusion et le partage de l'information. Il n'existe pas encore de cadre juridique international régissant l'utilisation des images produites par numérisation de collections patrimoniales, ainsi, les institutions sont soumises au droit spécifique du pays qui les abrite. En l'absence d'un encadrement global des pratiques, les utilisateurs -- et notamment la communauté de la recherche -- font face aux exceptions propres à chaque contexte national, qui représentent encore un obstacle. Une réflexion autour de ce cadre juridique international permettrait, en parallèle, d'harmoniser à une échelle globale les formats des données ouvertes\footcite{benhamouDroitAuteurMusees2016}, pour en faciliter la réutilisation par les institutions et par les projets.

    \subsubsection{Enjeux techniques : images et métadonnées}
	Le document numérique obtenu à la suite d'une processus de numérisation est, en réalité, constitué de plusieurs fichiers répondant à des besoins différents pour la mise en ligne. Chacun de ces éléments représente ses propres enjeux techniques, dont la gestion impacte la qualité et l'exploitabilité des images mises en ligne à la suite d'entreprises de numérisation.
	
	Les fichiers images issus d'un processus de numérisation peuvent avoir plusieurs formats\footnote{Un format décrit la manière dont les informations sont organisées dans un fichier.} aux propriétés similaires mais différentes. En premier lieu, il est de bonne pratique de privilégier un format ouvert\footcite{besseNumerisationMasseVers2019}, par opposition à un format propriétaire qui nécessite un logiciel spécifique pour être lu. Le choix d'un format ouvert permet d'assurer l'accessibilité des données produites. Les formats \acrfull{tiff} et \acrfull{jpeg} sont favorisés dans le contexte de numérisation des collections patrimoniales, parce qu'ils permettent notamment la compression des fichiers\footnote{La compression sans perte s'effectue par identification et suppression des redondances, c'est-à-dire des pixels identiques. Cette méthode n'est pas irréversible, il est possible de décompresser le fichier.} tout en maintenant la haute qualité des images\footcite{DigitalImages2020}. Les choix effectués par les institutions impactent la qualité des numérisations produites qui, sur un plus long terme, impacte aussi bien la préservation des données, l'expérience des utilisateurs qui consultent les images, et l'exploitabilité des fichiers par des projets. La question de la qualité des images est cruciale dans le cadre d'un projet de recherche en vision artificielle, puisque celle-ci a la possiblité d'impacter les performances de l'algorithme utilisé, et d'affecter négativement les interprétations du modèle\footcite{bergstromImageQualityComputer2023}.
	
	Les fichiers images sont systématiquement accompagnés de métadonnées de diverses catégories, qui visent à décrire et identifier le document. Bien qu'il n'existe pas de standard international obligatoire concernant les métadonnées de numérisation, la standardisation des fichiers de métadonnées selon certains standards spécifiques permet de fluidifier la communication et les échanges entre institutions, dans une perspective interopérable qui permettrait de sortir des silos de données\footnote{Un silo de données est un ensemble de données isolées ou difficilement accessibles. Dans le contexte des institutions patrimoniales et des établissements de recherche, dont les données se veulent publiques et partagées, les silos de données sont un obstacle à la mutualisation et à l'échange des ressources et informations.}. Les métadonnées sont de plusieurs types : descriptives\footnote{Pour identifier le contenu du document numérique, le rattacher au document original.}, administratives\footnote{Pour gérer les droits d'accès, préserver les informations techniques, suivre les modifications du fichier et garantir son intégrité.}, structurelles\footnote{Pour rattacher les fichiers entre eux et reconstituer la structure du document.} ; elles ont pour vocation de gérer tous ces aspects du document numérique. Ces métadonnées sont cruciales pour l'identification et la classification des documents numérisés, mais aussi pour permettre la recherche et la navigation à travers les collections. Malgré leur importance, les différentes institutions ont une gestion profondément hétérogène de leurs métadonnées, sans uniformisation nationale ou internationale, et avec une disparité profonde entre les différents types d'institutions patrimoniales.
	
	Pour pallier à cette disparité dans la gestion des métadonnées et au manque de standards pour le partage des images en ligne, des initiatives internationales ont vu le jour, visant à mettre fin à cette gestion en silo des données des institutions patrimoniales pour faciliter les échanges et la mise à disposition des données dans une optique de libre accès.
            
                \section{\label{stardardIiif}Le standard IIIF}
                    % Le standard IIIF

\subsection{IIIF et les données ouvertes}
    \subsubsection{Sous-sous-section 1}

    
    \subsubsection{Sous-sous-section 2}

    
    \subsection{IIIF, un modèle universel ?}
        \subsubsection{Sous-sous-section 1}


        \subsubsection{Sous-sous-section 2}


        \\
        
        Les sources iconographiques sont soumises à un ensemble de restrictions, du point de vue du format, des métadonnées ou des droits, qui manquent encore d'une uniformité internationale et entre institutions qui rendrait fluide le partage de ces ressources sur Internet et entre les projets de recherche. Dans un projet impliquant l'utilisation d'algorithmes de vision artificielle, qui repose alors sur le traitement d'un volume important d'image, la mise en ligne de ces documents est un enjeux crucial, sur lequel repose d'une part la possibilité de constituer un corpus exploitable, ainsi que la publication des résultats du projet, qui peuvent également prendre la forme d'images numériques. Dans une optique de science ouverte, l'utilisation de standards et d'outils tels que ceux développés par le consortium \iiif permet d'assurer le partage d'images et de données respectant les mêmes formats, exploitables avec des outils libres, et d'avancer vers une abolition des silos de données, qui faciliterait notamment la construction de corpus massifs pour l'apprentissage machine. 
        \clearemptydoublepage
        
        \chapter[Corpus historiques et jeux de données]{Corpus historiques et jeux de données pour l’apprentissage machine}
        
        La vision artificielle et l'apprentissage machine permet le traitement de corpus d'images massifs par des méthodes quantitatives qui permettent aux historiens de traiter un volume de données bien plus important qu'une approche manuelle, ouvrant ainsi la voie à de nouvelles approches. Cette partie revient sur les bonnes pratiques à mettre en place afin d'assurer la pertinence de l'utilisation de ces outils, et d'en faire des traitements efficaces en accord avec les ambitions des projets.
        
                \section{Dimensions et cadre}
                    % Dimensions et cadre

\subsection{La tentative de l’exhaustivité}
    \subsubsection{Représentativité et pertinence}
L'approche computationnelle de corpus de sources historiques permet le traitement d'un volume d'images inenvisageable sans automatisation, qui redéfinit ainsi le rôle des chercheurs comme les méthodes de constitution d'un corpus, pour prendre en compte cette quantité de données qu'il est possible de considérer à l'aide d'outils numériques\footcite{klinkeBigImageData2016}. L'apprentissage profond permet d'envisager une redéfinition des méthodes des historiens et historiens de l'art, en intégrant une part d'automatisation dans le traitement des sources iconographiques, permettant ainsi d'envisager l'exploration de corpus bien plus vastes\footcite{moiraghiExplorerCorpusImages2018} par des méthodes quantitatives\footcite{klinkeBigImageData2016}.

Dans le cadre du projet \eida, ces possibilités ouvertes par l'intégration de techniques de vision artificielle dans le traitement du corpus permet ainsi d'envisager des bornes larges, définies précédemment, sur le plan géographique comme chronologique, tout en assurant un traitement d'un grand nombre de sources -- plusieurs centaines ou milliers d'images -- pour chaque contexte étudié. 

Le projet \vhs s'appuie également grandement sur la vision artificielle appliquée à la détection de similarité, dans le cadre d'une étude de la circulation des savoirs scientifiques par le bais de l'illustration : les corpus constitués pour le projet ont donc été pensés pour offrir un regard pertinent sur ces questions, et sélectionnés avec une volonté de représentativité géographique, chronologique et thématique\footcite{Corpus}. Leur analyse portera ainsi sur un ensemble de plus de 10000 images d'animaux, de plantes et de minéraux.

Ainsi, l'intégration au cadre des projets de méthodes de vision artificielle permet d'envisager l'étude de corpus larges, vastes, et de porter un regard sur des zones géographiques larges, sur des périodes étendues, avec la possibilité de traiter des corpus massifs par des approches quantitatives développées en collaboration avec les historiens.

    \subsubsection{Données d'entraînement}
Au-delà de la pertinence du corpus pour un projet de recherche donné, il est nécessaire, pour l'entraînement d'un modèle de vision artificielle, de prévoir un fragment du corpus dédié à la constitution d'un ou de plusieurs jeux de données d'entraînement\footnote{Un jeu de données d'entraînement comptant en général plusieurs centaines d'exemples, il est possible, à cette étape, d'évaluer la pertinence véritable du développement d'un modèle dédié au traitement du corpus global du projet. \cite{strienComputerVisionHumanities2022}}. La représentativité est un élément clé de ces corpus restreints, qui doivent à leur échelle comporter suffisamment de cas d'études différents pour englober les situations rencontrées dans le corpus global, et ainsi produire un modèle apte à appréhender toutes les situations rencontrées dans un contexte d'inférence. Un dialogue entre les chercheurs et les ingénieurs est alors nécessaire, pour établir les besoins d'un point de vue technique comme d'un point de vue scientifique, afin d'obtenir en finalité un outil performant, apte à traiter le corpus du projet. Les chercheurs ayant une connaissance scientifique de la typologie des sources rencontrées dans le cadre du projet, ils sont ainsi à même d'estimer la diversité des images que rencontrera le modèle, et donc de construire un jeu de données d'entraînement représentatif et pertinent, en accord avec les bornes définies du projet.

Le projet \eida, dont le cadre géographique s'étend de l'Europe à l'Asie, du \viii au \xviii siècle, est ainsi vaste en termes de sources historiques et de diagrammes représentés. Les jeux de données d'entraînement doivent être représentatifs d'un point de vue thématique, en proposant des diagrammes aux formats et apparences diverses, en accord avec les typologies variées retrouvées dans le corpus ; mais ceux-ci doivent aussi prendre en compte la diversité des supports. Le corpus d'\eida comporte en effet des sources manuscrites comme des sources imprimées, qu'il est donc nécessaire de traiter conjointement. Il a été envisagé, pour des questions de performance, de produire un modèle de vision entraîné spécifiquement sur des sources manuscrites, et un second modèle entraîné sur des sources imprimées, afin de dissocier totalement le traitement de ces deux supports dans l'application finale\footnote{L'entraînement des modèles de détection d'objet n'ayant pas encore été effectué, il est actuellement impossible de tirer des conclusions quant à la pertinence de ce choix.}. Dans un corpus aussi varié linguistiquement que celui du projet \eida, il est également crucial de sélectionner un jeu de données d'entraînement diversifié du point de vue des langues, pour s'assurer de l'efficacité du modèle sur des sources aux provenances diverses : nous constatons en effet que les premières détections faites avec des modèles de détection pré-entraînés\footnote{Nous traitons de ces modèles dans la partie II.1.2.} sont satisfaisantes sur des sources latines ou grecques, mais qu'ils performent peu efficacement sur des sources chinoises, où chaque idéogramme est détecté comme une image.

Ainsi, l'automatisation d'une partie de la chaîne de traitement des sources permet, de construire un corpus de recherche vaste en offrant l'opportunité d'analyser un volume massif de sources historiques : ces possibilités sont particulièrement intéressante dans le cadre de projets de recherches inscrits dans l'histoire de la circulation des idées, des images et des théories, puisque les corpus construits peuvent être représentatifs de multiples cadres géographiques, temporels et thématiques, sans craindre un surplus de sources à traiter. Pour construire des outils pertinents pour le traitement de ces corpus massifs, il est cependant nécessaire d'avoir un regard global sur leur contenu, pour reproduire à moindre échelle leur représentativité dans des jeux de données d'entraînement, nécessaires au développement de modèles de vision artificielle pertinents pour des sources étudiées.
    
    \subsection{Automatiser le traitement de corpus massifs}
        \subsubsection{Possibilités de la \cv}
La vision artificielle offre, pour le traitement de ces larges corpus, la possibilité d'automatiser certaines étapes spécifiques qui interviennent en parallèle d'étapes d'analyse par les chercheurs. Il n'existe ainsi pas de \textit{workflow} totalement automatisée, où l'historien n'interviendrait pas dans le traitement des sources étudiées. En effet, l'intégration de la vision artificielle aux méthodes des historiens n'existe que conjointement à des interventions humaines, qui assurent la pertinence des traitements effectués et apportent une analyse nécessaire.

L'une des premières applications de la vision par ordinateur exploitée dans les projets étudiés est la détection d'objets dans les images : pour le traitement de sources numérisées, la détection d'objets présente des applications diverses. Il s'agit, le plus souvent, de la première étape d'une chaîne de traitement ; permettant, par exemple, dans l'étude de sources manuscrites ou imprimées contenant du texte et des illustrations, d'extraire les images présentes dans ces sources\footcite{buttnerCorDeepSacroboscoDataset2022}. Dans les projets \eida et \vhs, cette étape de détection des images dans les ouvrages est faite dans une optique de segmentation des pages, pour en extraire les illustrations qui seront ensuite traitées par d'autres algorithmes de vision, tels que des algorithmes de détection de similarité ou de vectorisation. D'autres projets, cependant, proposent dès l'étape de détection une classification des illustrations par typologie, pour permettre une première analyse et un premier regard sur la place de ces objets détectés dans les sources étudiées : l'application \textit{Cor}Deep\footcite{CorDeep}, développée par le Max Planck Institute for the History of Science en partenariat avec BIFOLD pour l'extraction d'éléments visuels dans les sources historiques, propose de classifier les illustrations détectées en quatre catégories\footnote{\textit{Content Illustrations}, \textit{Initials}, \textit{Decorations} et \textit{Printer's Marks}.}, appliquant dès la détection un premier traitement pour l'analyse des illustrations par les chercheurs, qui peuvent ainsi plus aisément naviguer des corpus de numérisations d'ouvrages par le biais des images.

La détection de similarité compte parmi les applications les plus directes du \dl\footcite{moiraghiExplorerCorpusImages2018} : il est possible, à l'aide d'un modèle entraîné sur un jeu de données restreint, d'effectuer sans supervision des comparaisons entre toutes les images du corpus, pour les réunir en séries ou groupes qui seront ensuite analysés par les historiens. Le développement d'un tel outil exemplifie les possibilités offertes par l'intelligence artificielle pour la navigation de gros corpus : elle permet de classifier les images en vue de leur étude en constituant des séries iconographiques aux caractéristiques visuelles similaires, selon un score de similarité calculé par l'algorithme. Dans le cas d'une étude des circulations des illustrations scientifiques, la constitution de ces séries permet aux chercheurs d'effectuer une analyse sur l'évolution des théories scientifiques, par le regroupement d'images de traditions différentes présentant des éléments similaires. L'automatisation de cette étape permet ainsi de tracer des parallèles entre un grand nombre d'image, en lançant la détection sur des corpus très larges qu'il serait difficile de traiter manuellement.

La \cv permet également d'envisager des traitements pour l'édition, et particulièrement dans le cas d'objets scientifiques tels que les diagrammes, dont la mise en forme est un questionnement à part entière pour l'édition des textes qu'ils illustrent. La détection de contours (ou \textit{edge detection}) est une étape fondamentale pour le traitement des images en vision artificielle : combinée à des méthodes de détection des lignes\footcite{linComprehensiveReviewImage2023}, il devient envisageable d'automatiser la transformation d'images au format .tiff ou .jpeg en objets plus aisément manipulables, et notamment la transformation vers le \svg, qui fait alors des diagrammes des objets édités et éditables, exploitables par les chercheurs en tant qu'objets numériques.

        \subsubsection{Diversité des traitements, diversité des données}

Pour ces divers traitements exploitant la vision artificielle, l'étape d'entraînement est nécessaire pour l'obtention d'un modèle efficace sur les données du projet, et pour que celui-ci performe en accord avec les tâches qui lui sont demandées. L'entraînement d'un modèle nécessite généralement plusieurs étapes\footcite{strienComputerVisionHumanities2022}, et par conséquent autant de jeux de données qu'il y a d'étapes : il est de bonne pratique de prévoir un jeu de données initial, et d'évaluer les performances du modèle après ce premier entraînement. À partir de ces résultats, il devient possible d'adapter les jeux de données suivants, pour pallier aux faiblesses constatées. De plus, si le format des données d'entraînement nécessaires varie en fonction des tâches effectuées par le modèle\footnote{Les cas varient en fonction du format des données d'entrée et de sortie : images au format .jpeg et annotations au format .txt pour la détection, images au format .jpeg et images au format .svg pour la vectorisation, etc.}, les exigences en termes de volume et de représentativité varient également. 

Pour des tâches classiques de vision artificielle, telles que la détection d'objets, des modèles pré-entraînés performants existent en libre accès, qui ne nécessitent donc pas autant de travail d'entraînement qu'un modèle créé dans son entièreté pour un projet. Des jeux de données en accès libre sont également disponibles pour l'entraînement de modèles pour la détection automatique, tels que ImageNet\footcite{ImageNet}, qui bien qu'ils ne soient pas adaptés à des sources historiques, permettent d'effectuer un premier entraînement d'un modèle sans mobiliser les moyens nécessaires à la création d'un jeu de données. Les algorithmes de détection de similarité ou de vectorisation, cependant, ne sont pas aussi développés que ces algorithmes de détection d'objets, et nécessitent donc un volume de données plus important, que les projets se doivent de fournir pour la création de modèles performants.

Ainsi, pour un projet d'humanités numériques faisant appel aux méthodes de la vision artificielle, il est nécessaire d'apporter une réflexion, en premier lieu, sur la pertinence de l'usage de l'\ia pour le traitement du corpus -- pertinence souvent relative au volume de données à traiter. Il est ensuite nécessaire, au-delà des sources du projet, de penser les données pour le \dl, en prenant en compte l'importance d'avoir des jeux de données qualitatifs et représentatifs pour la création d'un modèle efficace pour l'automatisation des tâches souhaitées. Les formats, typologies, volumes de ces jeux de données d'entraînement, souvent multiples, est ainsi à considérer en collaboration entre les équipes d'ingénierie et de recherche, pour prendre en considération les besoins vis-à-vis du contenu du corpus, tout en ne perdant pas de vue les besoins techniques liés à l'entraînement d'un modèle de vision par ordinateur.

            
                \section{\label{objectifsPossibilites}Objectifs scientifiques et possibilités numériques}
                    % Annotations et qualité des données

\subsection{Dialogue entre chercheurs et ingénieurs}
    \subsubsection{Concilier les besoins}
L'intégration de traitements employant l'\ia a des projets de recherche en humanités implique une redéfinition des méthodes, pour adapter les pratiques des chercheurs aux nouveaux besoins soulevés par ces techniques. Dans la section précédentes, nous avons établi les nécessités et possibilités liées aux corpus de recherches, mais il est également crucial de prendre en compte les échanges entre les acteurs divers impliqués dans de telles initiatives. Les projets étudiés dans le cadre de ce mémoire -- notamment les projets \eida et \vhs -- présentent une architecture similaire en termes d'équipes impliquées : il est, d'une part, une équipe d'historiens\footnote{Composée de chercheurs permanents, de chercheurs affiliés, de chercheurs postdoctorants et de doctorants.} menant des recherches sur les thèmes étudiés, travaillant en collaboration avec une équipe de chercheurs en vision artificielle\footnote{Dans les deux cas cités, il s'agit de l'équipe dirigée par Mathieu Aubry, appartenant au groupe de recherche \imagine.}. Les équipes d'histoire des sciences comptent, dans le cas de \vhs et \eida, un ou plusieurs ingénieurs, chargés de développement spécifiques et d'établir la communication avec les équipes de vision.

Les deux équipes de recherche impliquées dans les projets font ainsi face à des besoins variés, qui ne sont pas nécessairement alignés. La communication établie entre ces deux pôles permet de clarifier les besoins en termes de traitement des sources historiques, par opposition aux besoins en termes de techniques, et particulièrement de définir les limites de ces méthodes et techniques numériques appliquées à l'histoire. Il se dessine alors une tension entre les attentes vis-à-vis des modèles de vision, et les possibilités réelles de la technique, qu'il est crucial de prendre en compte pour le développement d'outils à la fois performants et pertinents.
    
    \subsubsection{Utilité pour le \ml et intérêt historique}
    
Ces tensions entre les besoins se manifestent dès les premières étapes de développement d'un modèle de vision artificielle : dans le cas du projet \eida, porté sur la détection de diagrammes dans les manuscrits, la confrontation entre l'aspect binaire de la détection et les nuances que doivent présenter l'approche historique se présente dès l'étape de constitution d'un premier jeu de données d'entraînement. En effet, cet aspect binaire du modèle, qui s'appuie sur l'opposition entre ce qui est un diagramme et ce qui n'en est pas un -- et plus largement, ce qui est une objet à détecter et ce qui n'en est pas un -- ne peut prendre en compte les cas limites qui se présentent systématiquement avec des sources historiques provenant d'un corpus aussi large. Se pose ainsi la question de la conciliation de ces deux exigences visiblement opposées : il est nécessaire de décider, par une discussion entre l'équipe de vision artificielle et l'équipe d'histoire des sciences, de ce qui est bel et bien un diagramme, selon les possibilités techniques et les besoins historiques.

Une distinction se crée entre les pratiques utiles pour le \dl et les éléments intéressants pour les historiens. L'objectif, pour les deux équipes, est en effet de construire un outil utile aux résultats exploitables ; les projets ont donc pour devoir de prendre en compte ces besoins divergents, et d'établir un juste milieu entre les possibilités offertes et les objectifs initiaux. 

Ainsi, dans cette situation, un équilibre est défini entre performances du modèle et pertinence historique, en prenant en compte le fait qu'il existe, après la détection, une intervention humaine : aucune solution n'étant pleinement satisfaisante, il convient de décider de celle qui permet de concilier les possibilités techniques et les objectifs scientifiques, afin de créer un outil qui produit des données exploitables, et qui automatise de manière pertinente le traitement des sources. Dans cette démarche, il est nécessaire de donner un cadre strict aux pratiques pour assurer une uniformisation des données, premier critère pour le développement d'outils numériques efficaces aux résultats cohérents.
    
    \subsection{Définir les pratiques}
        \subsubsection{Traiter les cas limites}
Il existe autant de manière de gérer ces situations qu'il existe de projet : les pratiques sont à définir en accord avec les besoins de chaque projet, sans qu'il n'existe de solution universelle. Les normes à appliquer sont à établir par un dialogue prenant en compte les ambitions techniques et scientifiques de chaque projet, ainsi que les spécificités des sources étudiées, qui présenteront des caractéristiques spécifiques propres à chaque corpus.

En amont de la constitution du jeu de données pour l'entraînement du modèle de détection des diagrammes astronomiques du projet \eida, cette discussion entre l'équipe de l'Observatoire et celle du laboratoire \imagine a mené à la décision d'annoter tout type de diagramme, ainsi que les illustrations retrouvées dans les manuscrits, même lorsqu'elles ne correspondent pas aux objets recherchés. En effet, l'exclusion d'illustrations telles que les miniatures dans les données d'entraînement -- même si ces dernières ne correspondent pas aux objectifs scientifiques du projet -- pourrait produire un modèle aux détections lacunaires. La solution choisie l'a été en prenant en considération le fait qu'il est plus aisé pour un historien qui analyserait les images détectées d'exclure les illustrations qui ne sont pas des diagrammes, plutôt que de rechercher dans la numérisation des diagrammes manqués par un modèle aux performances insatisfaisantes.

La création d'un premier jeu de données pour l'entraînement d'un modèle dédié à la détection des illustrations dans le corpus de \vhs a permis de mettre en lumière un certain nombre de cas limites, ou \textit{edge cases}, qui correspondent techniquement aux pratiques d'annotation définies par le projet, mais qui néanmoins impacteraient négativement les performances du modèle. Ainsi, nous voyons dans les annotations réalisées par les chercheurs de \vhs des pratiques à modifier (fig. \ref{fig:edge_cases_vhs}). Dans le premier cas, toutes les illustrations de la page n'ont pas été annotées : seule l'image principale de la planche zoologique est annotée comme un objet à détecter, tandis que les plus petites images sont délaissées. Comme mentionné précédemment, l'aspect manichéen de la détection par vision artificielle de permet pas de hiérarchiser ainsi les images considérées intéressantes pour la recherche, et celles qui ne le sont pas, il est donc nécessaire d'annoter l'ensemble des illustrations pour ne pas impacter négativement les performances du modèle, et risquer une détection lacunaire. Dans les deuxièmes et troisièmes cas, les images annotées n'auraient pas dû l'être, car si la mauvaise qualité de la numérisation ou l'aspect fragmenté du document n'empêche pas un observateur humain d'identifier les illustrations comme telles, il est une part de reconstitution des pans manquants de l'image qui tient de l'interprétation, et qui ne peut donc être attendue d'un modèle de vision artificielle. L'annotation de ce type d'objets pour l'entraînement du modèle risquerait ainsi d'altérer ses performances, en le poussant à détecter comme des illustrations des éléments qui n'en seraient pas.

	\begin{figure}[h]
	\begin{subfigure}{0.37\linewidth}
		\centering
		\includegraphics[height=4.5cm]{images/vhs_edge_case1.png}
		\subcaption{Annotation incomplète des illustrations}
	\end{subfigure}
	\hspace{1pt}
	\begin{subfigure}{0.30\linewidth}
		\centering
		\includegraphics[height=4.5cm]{images/vhs_edge_case2.png}
		\subcaption{Annotation d'un document mal numérisé}
	\end{subfigure}
	\hspace{1pt}
	\begin{subfigure}{0.25\linewidth}
		\centering
		\includegraphics[height=4.5cm]{images/vhs_edge_case3.png}
		\subcaption{Annotation d'un document lacunaire}
	\end{subfigure}
	\caption{Cas limites rencontrés lors de l'annotation des images du projet \vhs}
	\label{fig:edge_cases_vhs}
\end{figure}

Ces décisions quant à l'annotation sont ainsi prises suite à un dialogue entre les équipes de chercheurs en vision et d'historiens, pour trouver un juste équilibre entre attentes et possibilités réelles. Des choix sont faits pour tirer parti des méthodes de vision artificielle, tout en acceptant que leurs résultats ne seront jamais identiques à ceux d'un travail manuel : les décisions doivent ainsi être prises pour exploiter de manière pertinentes ces techniques. Les pratiques définies par ces discussions entre les équipes mènent à des normes d'annotation : des spécifications sont alors rédigées pour les établir, et s'assurer de leur application uniforme, pour la création de données cohérentes.

        \subsubsection{Guides des pratiques : établir les normes}

Pour s'assurer de l'application uniforme des normes d'annotation, il est de bonne pratique de rédiger une documentation à destination des chercheurs, à laquelle chacun peut se référer lors des étapes de création des données d'entraînement, qui constituent en elles-mêmes un travail considéré comme travail de recherche, puisqu'il requiert une connaissance et une compréhension des sources. Ainsi, les données produites suivent les mêmes règles, et cette cohérence assure une uniformité de jeu de données fourni au modèle de vision : cette uniformité permet, s'il une nouvelle étape d'entraînement est nécessaire, de corriger plus facilement les choix ayant mené à des résultats non-satisfaisants.

Les chercheurs du projet \eida se sont vus fournir un \textit{Annotation Tutorial} en anglais, rédigé par Ségolène Albouy, cheffe de projet numérique, et Jil Le Bois, stagiaire de licence. Cette documentation fait suite à la mise en place d'une chaîne de traitement automatique des sources déposées par les chercheurs sur l'application du programme, qui produit une première détection que les chercheurs ont ensuite pour devoir de corriger. Le tutoriel qui a été communiqué contient des explications sur l'annotation dans l'application, avec un pas-à-pas détaillant chaque étape de ce travail, ainsi qu'une liste de cas pratiques établie en communication avec les historiens et l'équipe du laboratoire \imagine.

Les cas pratiques établis répondent -- le plus souvent -- à des questions posées par les historiens à l'équipe d'ingénierie à l'occasion d'ateliers ou de séminaires, qui permettent de souligner au préalable les possibles doutes qui seront rencontrés lors de l'annotation. L'\textit{Annotation Tutorial} permet de garder trace de ces questionnements, et d'y apporter une réponse dont l'application sera systématique. La production de cette documentation, dont la responsabilité incombe aux ingénieurs d'étude, en tant que vecteurs de la communication entre les équipes d'histoire et de vision, pour donner à chaque parti impliqué la possibilité de se référer à un document qui rappelle les décisions prises. Cette possibilité de se référer à un document exhaustif est importante pour les étapes d'annotation, puisqu'elle garantit des données uniformes, mais aussi lors des étapes postérieures, notamment si d'autres entraînements sont nécessaires pour affiner les performances du modèle.

        \\
        
        Le \dl et la vision artificielle permettent aux projets de recherche en histoire et en histoire de l'art d'envisager de nouvelles approches des sources, à l'aide de traitement automatisés qui offrent de nouvelles méthodes de navigation de corpus d'images massifs : de la détection à l'édition, les outils produits redéfinissent les étapes de traitement des sources, et il est ainsi nécessaire d'intégrer aux pratiques des chercheurs des méthodes spécifiques à ce type d'approche, et notamment pour la création de jeu de données d'entraînement, à la base du développement de tout modèle de \ml. Cette intégration passe notamment par la rédaction de documentation, ainsi qu'un dialogue entre les équipes de recherche en vision artificielle et les équipes d'historiens. Ce dialogue permet d'établir les besoins de chacun, et de développer des outils techniques qui répondent aux besoins scientifiques de manière pertinente, en développant des modèles qui, malgré leurs limites, rejoignent aux mieux les attentes des sciences historiques.
        \clearemptydoublepage


    \part{De l’image à l’objet : intégrer l’apprentissage profond au traitement des sources historiques}
        \chapter[L'apprentissage profond]{Principes et utilisation de l’apprentissage profond}
        
        L'apprentissage profond est un sous-domaine de l'apprentissage automatique basé sur l'apprentissage de couches successives de représentation. Le nombre de couches définit la profondeur du modèle : de nos jours, l'apprentissage profond compte plusieurs dizaines à plusieurs centaines de couches, qui apprennent toutes automatiquement à l'aide de données d'apprentissage\footcite{cholletApprentissageProfondAvec2020a}. Cette approche est au cœur des modèles de vision artificielle dont nous parlons dans ce mémoire, qui reposent sur des réseaux de neurones qui constituent ces couches superposées permettant un apprentissage des représentations à partir de données fournies.
        
                \section{Réseaux de neurones et \textit{computer vision}}
                    % Réseaux de neurones et computer vision

	\subsection{Des \og neurones \fg pour le \dl}
	Les modèles de vision artificielle pour des tâches telles que la détection d'objets, la détection de similarités ou la détection de lignes reposent souvent, en leur cœur, sur des réseaux de neurones plus ou moins profonds. Un réseau de neurones artificiels est un modèle paramétrique plus ou moins complexe\footcite{azencottIntroductionAuMachine2018} composé d'une couche d'entrée recevant les données brutes, puis d'une ou de plusieurs couches de calcul traitant les données en se corrigeant mutuellement, et d'une couche de sortie proposant une prédiction à partir des données d'entrée et des calculs effectués. Un réseau de neurones est dit \og profond \fg lorsqu'il compte un nombre suffisant, variable, de couches\footcite{azencottIntroductionAuMachine2018}. La prédiction se fait à l'aide d'un système de poids, qui correspondent aux paramètres d'une couche pour la transformation qu'elle applique aux données d'entrée\footcite{cholletApprentissageProfondAvec2020a}, et qui a pour objectif de calculer la prédiction la plus juste, c'est-à-dire d'associer l'entrée avec une cible\footnote{Pour un algorithme de détection d'images dans les manuscrits, l'entrée est l'image de la page, et la cible est l'étiquette \og Illustration \fg qui sera attribuée à la zone de l'image où une illustration est détectée.} par une série de transformations. L'algorithme corrige ses poids en cas de prédiction fausse : il s'agit de l'apprentissage, qui signifie donc la recherche d'un ensemble de valeurs pour les poids de toutes les couches du réseau de sorte que les résultats obtenus soient satisfaisants.
    
    \subsection{Réseaux de neurones à convolution}
    Les modèles mis en avant dans ce mémoire de recherchent reposent tous plus précisément sur des réseaux de neurones à convolution, ou \cnn, développés pour reconnaître des motifs visuels dans des images avec un minimum de traitements appliqués. Ces derniers sont favorisés pour les tâches de vision par ordinateur, et pour les tâches de perception en général\footcite{cholletApprentissageProfondAvec2020a} Ces réseaux sont composés de deux types de neurones agencés en plusieurs couches : les neurones de traitement, dédiés à traiter chacun une portion de l'image, et les neurones de mise en commun des sorties, dits de \textit{pooling}\footcite{goodfellowDeepLearning2016}. Les couches de convolution\footnote{Une couche de convolution est une couche constituée de copies d'un même neurone qui ne prend en compte qu'une partie le l'entrée.} ont pour spécificité d'être basées sur des fragments qui ne représentent que quelques pixels des images d'origine\footnote{Ces motifs locaux peuvent être les bords, les textures, et d'autres éléments. \cite{cholletApprentissageProfondAvec2020a}} : elles apprennent des motifs locaux, à l'inverse des couches entièrement connectées qui apprennent des motifs globaux. Les \cnn ont besoin de moins d'exemples d'apprentissage, parce qu'ils ont la capacité d'apprendre un motif et de le reconnaître quelle que soit sa position. Les \cnn requièrent donc un volume de mémoire moins important, pour une efficacité supérieure\footcite{goodfellowDeepLearning2016}. 
    
    Les \cnn ont la capacité d'apprendre des motifs locaux, puis d'apprendre dans les couches suivantes des motifs plus grands qui en découlent, apprenant ainsi des concepts visuels de plus en plus abstraits, et de plus en plus complexes\footcite{cholletApprentissageProfondAvec2020a} : l'intérêt dans le cadre d'un apprentissage pour la vision artificielle réside donc dans cette possibilité d'apprendre des motifs invariants par translation\footnote{\og  L'invariance par translations est une propriété fondamentale partagée par la quasi-totalité des opérations de traitement d'images. Elle exprime le fait qu'une information visuelle sera traitée de la même façon, quelle que soit sa localisation dans l'espace. En effet, si on bouge le capteur, chaque objet sera déplacé dans l'image, mais devra être traité de la même façon que précédemment. \fg \cite{ronseInvarianceParTranslations}} et spatialement hiérarchiques, deux caractéristiques fondamentales du monde visuel\footcite{cholletApprentissageProfondAvec2020a}.


            
                \section[Modèles de vision \textit{off-the-shelf}]{Modèles de détection \textit{off-the-shelf} : outils libres pour l'extraction d’objets}
                    % Modèles de vision off-the-shelf : outils libres pour la détection d’objets

\subsection{YOLOv5, \docex et autres modèles}
    \subsubsection{Utilisation d'un modèle pré-entraîné}
	Il est courant, pour la vision artificielle, d'utiliser un modèle pré-entraîné sur un grand jeu de données peu spécifique, et de l'affiner à partir d'un plus petit jeu de données correspondant précisément aux attentes et applications du modèle dans le projet\footnote{Nous nous focaliserons spécifiquement sur les modèles de détection et de classification, dont la question de l'entraînement a été abordée lors du stage.}. En effet, un modèle entraîné sur un jeu de données de base suffisamment large et généraliste apprend des caractéristiques qui peuvent être appliquées à l'ensemble du monde visuel, et peut ainsi servir de modèle générique pour de nombreux problèmes de vision par ordinateur\footcite{cholletApprentissageProfondAvec2020a}, même si ces derniers sont éloignés de la tâche initiale. La portabilité des caractéristiques apprises rend ainsi l'usage du \dl pertinent dans le cadre de projets de recherche sur des sources historiques, puisqu'elle assure ainsi l'efficacité des modèles créés malgré les possibles limites des jeux de données disponibles : il est ainsi possible d'entraîner un modèle sur un jeu de données limité, qui correspond à la réalité matérielle des sources étudiées. 
	
	\subsubsection{ImageNet} 
	ImageNet\footcite{ImageNet} est une initiative ayant pour objectif de fournir en accès libre un vaste jeu de données images pour la recherche en vision artificielle. Cette initiative est née du besoin crucial de données d'entraînement et de validation pour le \ml, et particulièrement pour la classification, tâche de base de la vision par ordinateur, qui requiert un volume important d'images pour l'obtention de bonnes performances. Le jeu de données ImageNet compte plus d'un million d'images, catégorisées selon des concepts récupérés du projet WordNet qui vise à répertorier et classifier le contenu sémantique et lexical de la langue anglaise. Chaque image du jeu de données est annoté manuellement pour indiquer les objets présent dans l'image : ImageNet est un projet participatif, ce qui permet d'assurer la continuité de son enrichissement. Le projet ne possède pas les droits des images du jeu de données, formulant simplement une liste d'images disponibles sur le Web pour chaque concept WordNet : elles sont mises à disposition des projets à visée non-commerciale, pour assurer aux chercheurs la possibilité de mener des recherches en ayant accès à un jeu de données d'entraînement riche et vaste, permettant de développer des modèles performants avant un entraînement plus spécialisé.
    
    \subsubsection{\yolov : un modèle de pointe pour l'extraction d'objets}
	\yolov\footcite{ultralyticsUltralyticsYOLOv8Docs} est la cinquième version du modèle de détection d'objets et de segmentation d'images \yolo, développé à l'Université de Washington Joseph Redmon et Ali Farhadi et lancé en 2015. \yolov, publié en 2020, est un modèle réputé pour sa rapidité d'exécution et sa précision\footcite{buttnerCorDeepSacroboscoDataset2022}, pré-entraîné sur le jeu de données MS COCO\footcite{COCOCommonObjects}, dont l'implémentation se veut aisée : \yolov, contrairement à ses prédécesseurs, est directement implémenté dans PyTorch\footcite{PyTorch}, permettant une intégration facile à un environnement de développement, puisqu'il nécessite moins d'adaptation que les versions précédentes fonctionnant avec \textit{framework} Darknet\footcite{DarknetOpenSource}, basé sur le langage C\footcite{sharmaTrainingYOLOv5Object2022}. Le réseau neuronal de \yolo fonctionne en trois étapes : l'extraction de caractéristiques des données d'entrée, l'aggrégation de ces caractéristiques, puis la résolution du problème (dans ce cas précis, la détection des objets)\footcite{buttnerCorDeepSacroboscoDataset2022} Le modèle est disponible en accès libre, et constitue ainsi une base solide pour les projets souhaitant utiliser un modèle de détection d'objets : \yolov peut, en effet, être entraîné sur un jeu de données choisi, pour en affiner la détection et l'appliquer à des images plus spécifiques, telles que des pages d'ouvrages.
	
	Les projets \eida et \vhs entraînent, pour la détection d'illustrations dans les numérisations d'ouvrages, des modèles ayant pour base \yolov. Sans entraînement spécifique, en s'appuyant exclusivement sur le pré-entraînement fait avant la mise en ligne du modèle, les performances sont peu satisfaisantes : les jeux de données utilisés pour ce pré-entraînement, qu'il s'agisse d'ImageNet ou de MS COCO, sont en effet des jeux de données d'images réelles, faits pour l'apprentissage de la classification d'objets du quotidien, et ne sont donc initialement pas adaptés à la segmentation de pages de manuscrits ou à la détection d'illustrations. Il faut ainsi compter sur les propriétés de portabilité de l'apprentissage, qui assurent que ce pré-entraînement sur des images réelles permet d'accélérer le processus de développement du modèle, puisqu'elles permettent de s'intéresser immédiatement à l'entraînement à partir de données spécifiques, et réduisent le volume nécessaire. 
	
	\subsubsection{\docex : un modèle pour les sources historiques}
   	\docex est un modèle \textit{off-the-shelf} de détection d'objet, dédié spécifiquement aux documents historiques\footcite{monnierDocExtractorOfftheshelfHistorical2020}. Développé par Tom Monnier, ce modèle a pour vocation d'être un outil efficace et prêt à l'usage pour le traitement de documents numérisés, capable de détecter le texte et segmenter les lignes, et d'extraire les illustrations en détectant leurs contours précis\footnote{Contrairement à \yolov qui produit des annotations rectangulaires et ne gère pas les polygones.}. \docex est entraîné à partir d'images produites par un générateur de documents historiques synthétiques (fig. \ref{fig:syndoc}), qui promet de bonnes performances même sans \textit{fine-tuning} : l'utilisation de ce générateur automatique, appelé SynDoc\footnote{Les images générées par SynDoc sont composées aléatoirement à partir d'une sélection d'images de fonds (pages et contextes), auxquelles sont ajoutées une mise en page selon laquelle est disposé un contenu image et texte, puis du bruit. Tous ces éléments sont tirés d'un jeu d'images pré-établi (constitué de 177 images de pages, 15 contextes, plus de 8000 œuvres d'art provenant de WikiArt, des lettrines générées à partir d'une lettre aléatoire avec 91 fonts possibles, et des dessins, schémas et textes tirés d'articles aléatoires sur Wikipedia, avec plus de 400 fonts) et composent des pages aléatoires mêlant images, texte et bruit, avec un nombre vaste de possibilités qui égale les plus grands jeu de données d'entraînement. Ces pages ne nécessitent pas d'annotations manuelles, puisque chaque élément de contenu est pré-annoté.}, vise à répondre aux problématiques de création de jeux de données d'entraînement à partir de documents historiques, limités à la fois par le nombre de documents disponibles et les moyens humains que l'annotation d'images nécessite. SynDoc permet ainsi, dans la création d'un modèle, de combler l'absence d'un jeu de données de grande échelle d'images de documents historiques annotées\footcite{buttnerCorDeepSacroboscoDataset2022}.
   	
   	\begin{figure}[h]
   		\centering
   		\includegraphics[width=14cm]{images/syndoc.jpg}
   		\caption{Exemples tirés d'un jeu de données généré avec SynDoc}
   		\label{fig:syndoc}
   	\end{figure}
   	
   	Contrairement à \yolov, \docex est donc développé spécifiquement pour le traitement des sources historiques, en prenant en compte les spécificités de ces documents iconographiques : le modèle est ainsi adapté au traitement d'images de pages contenant du texte et des illustrations, et prévoit également des outils de traitement du texte, tels que la détection des lignes en prévision\footnote{La détection des lignes est une étape préliminaire de traitements tels que la reconnaissance optique de caractères (OCR) ou la reconnaissance de l'écriture manuscrite (HTR).}. \docex semble ainsi être le modèle à favoriser dans le cadre des projets mentionnés dans ce mémoire, cependant, les premières évaluations des modèles\footnote{Les deux modèles ont fait l'objet d'un entraînement préliminaire sur des sources historiques qui ne sont cependant pas celles des deux projets, pour un premier affinement de leurs performances.} sur les sources d'\eida et de \vhs semblent témoigner de performances équivalentes.
   	
   	En effet, les détections de diagrammes lancées sur des sources du corpus \eida en prévision de leur annotation par les chercheurs pour la constitution du jeu d'entraînement ont témoigné de performances prometteuses aussi bien pour \docex que pour \yolov  (fig. \ref{fig:performances_modeles}), avec des diagrammes non-détectés dans les deux cas : les performances de chacun des deux modèles varient en fonction des pages, et il sera donc nécessaire de les départager en évaluant plus précisément leurs performances en amont de l'entraînement, puis après un premier entraînement des deux modèles sur les sources du projet\footnote{Sur le projet \eida, les modèles ont été fournis par les chercheurs du laboratoire \imagine et l'ingénieur du projet \vhs, n'ont donc pas encore été évalués sur les sources astronomiques du projet. La vérité de terrain étant encore en cours de production à l'été 2023, l'évaluation sera faite à partir d'un fragment de ce jeu de données lorsqu'il sera disponible.}. En tant que modèles \textit{off-the-shelf} en libre accès, \yolov et \docex ont pour vocation d'être efficaces et performants même sans entraînement sur des données spécifiques, et permettent ainsi de mener de manière satisfaisante des tâches de détection et d'extraction sur des données personnelles. Cependant, pour obtenir de meilleures performances, il est préférable d'entraîner ces modèles \textit{off-the-shelf} sur des images tirées du corpus du projet, pour l'adapter et le spécialiser dans la résolution de ces problèmes particuliers.
   	
\subsection{Entraînement et \textit{fine-tuning} d'un modèle de détection}
    \subsubsection{Démarche et volume des données}
	L'entraînement d'un modèle sur un jeu de données spécifiques permet de préciser les tâches effectuées et d'obtenir un modèle adapté aux besoins spécifiques de chaque projet. L'utilisation d'un modèle pré-entraîné permet, comme mentionné précédemment, de pallier aux limites de jeux de données avec trop peu d'exemples, en exploitant les représentations apprises précédemment et en les spécialisant sur les données spécifiques du projet qui l'utilise.
	
	\begin{figure}[H]
		\begin{subfigure}{1\linewidth}
			\centering
			\includegraphics[width=10cm]{images/ms40_p28_docext.png}
			\subcaption{p. 28, \docex}
		\end{subfigure}
		\hspace{1pt}
		\begin{subfigure}{1\linewidth}
			\centering
			\includegraphics[width=10cm]{images/ms40_p28_yolov5.png}
			\subcaption{p. 28, \yolov}
		\end{subfigure}
		\begin{subfigure}{1\linewidth}
			\centering
			\includegraphics[width=10cm]{images/ms40_p45_docext.png}
			\subcaption{p. 45, \docex}
		\end{subfigure}
		\hspace{1pt}
		\begin{subfigure}{1\linewidth}
			\centering
			\includegraphics[width=10cm]{images/ms40_p45_yolov5.png}
			\subcaption{p. 45, \docex}
		\end{subfigure}
		\caption{Diagrammes détectés par les modèles \yolov et \docex dans deux pages du manuscrit hebr. Or. fol. 1054 de la Staatsbibliotek (Berlin).}
		\label{fig:performances_modeles}
	\end{figure}

	Les modèles de détection \textit{off-the-shelf} comme \yolov prévoient souvent un \textit{workflow} pour l'entraînement du modèle\footcite{sharmaTrainingYOLOv5Object2022} par des projets qui souhaiteraient le spécialiser. Ainsi, dans le cas de \yolov, l'entraînement étant prévu par les développements mis à disposition par Ultralytics, il est nécessaire pour les ingénieurs de prévoir des jeux de données d'entraînement adaptés au format requis par le modèle\footnote{Les projets \vhs et \eida emploient tous deux \yolov comme modèle, ou \docex intégré au \textit{workflow} développé par Ultralytics pour \yolov : la structure des jeux de données d'entraînement est donc la même pour les deux modèles.} : l'annotation par les chercheurs doit donc être effectuée en prenant en compte ces restrictions, et les outils développés par les ingénieurs pour l'annotation des images doivent avoir pour sortie des fichiers aux formats appropriés. Un document de spécification sur les formats d'image et d'annotation a été réalisé dans le cadre du projet \eida (Annexe \ref{YOLOv5Training}), pour garder traces des besoins techniques de l'entraînement, étape suivante du projet. 
	
	L'entraînement d'un modèle \yolov se fait à partir d'un \textit{dataset} d'images et d'étiquettes (\textit{labels}) : il s'organise ainsi en un dossier de fichiers image (au format .png ou .jpg), et un dossier de fichiers texte (au format .txt) contenant une liste d'objets. Chaque fichier texte correspond à un ou plusieurs fichiers image -- plusieurs dans le cas de numérisation d'ouvrages, notamment, qui se composent d'un fichier d'annotations par ouvrage numérisé et non par image -- et listent les objets détectés dans l'image, à raison d'une ligne par objet. L'objet est caractérisé par ses dimensions (selon un rectangle qui l'encadre) et ses coordonnées sur l'image. Par exemple, le fragment de ficher d'annotation suivant correspond aux quatre objets détectés sur la page 6 du manuscrit 5 (Alm. 1) de la Gurukul Kangri Haridwar Collection (fig. \ref{fig:annotation_ms143}) :
	
	\begin{lstlisting}
		6 ms143_0006.jpg
		243 1852 464 549
		168 7 763 618
		440 951 1174 1130
		919 47 851 826\end{lstlisting}

   	\begin{figure}[H]
		\centering
		\includegraphics[width=15cm]{images/ms143_p6.png}
		\caption{Objets détectés sur la page 6 du manuscrit 5 (Alm. 1) de la Gurukul Kangri Haridwar Collection }
		\label{fig:annotation_ms143}
	\end{figure}

	À partir de cet ensemble d'images et d'annotations, l'entraînement peut être effectué : \yolov prévoit un script pour l'entraînement d'un modèle, dont seuls les paramètres doivent être modifiés pour l'adapter aux besoins du projet. Ainsi, les modèles de détection \textit{off-the-shelf} tels que \yolo ou \docex prévoient la possibilité d'un entraînement, simplifié par la mise à disposition d'outils permettant d'effectuer cette étape en écrivant peu de code. Cette volonté de rendre accessible des outils adaptables, construits comme un socle solide pour des objectifs variés, permet l'intégration de la vision artificielle à des projets divers, sans demander de ces derniers de créer ou d'entraîner de zéro des réseaux de neurones ou des modèles de détections. La détection étant, en effet, une tâche de base de la vision par ordinateur, il est peu pertinent d'allouer des ressources pour reproduire des techniques déjà appliquées par de nombreux projets, et les outils \textit{off-the-shelf} offrent donc la liberté d'envisager la vision artificielle comme élément d'une chaîne de traitement des sources en allégeant les besoins et ressources -- notamment humains et temporels -- que demandent le développement et l'application de ces techniques. 

    \subsubsection{Sur-ajustement et diversité des exemples}
	
	L'utilisation d'un modèle pré-entraîné permet d'obtenir un modèle de détection fonctionnel malgré une faible quantité de données disponible, puisque celui-ci a la capacité de généraliser les caractéristiques apprises à partir des données de pré-entraînement et de les appliquer aux données spécifiques du projet. Un trop faible volume de données présente en effet le risque de produire un modèle peu performant à cause du sur-ajustement (ou \textit{overfitting})) : on parle de sur-ajustement lorsqu'un modèle est bien ajusté aux données d'entraînement, mais incapable de généraliser face à de nouvelles données\footcite{cholletApprentissageProfondAvec2020a}. 
	
	Il est crucial de prévoir, lors de la constitution du jeu de données d'entraînement, une portion dédiée à la validation : on considère généralement que 20\% du jeu de données total est un volume suffisant pour cette étape. Les données de validation sont des données -- dans le cas de la détection, des images et annotations produites par des humains -- qui n'ont jamais été vues par le modèle et n'ont donc pas été utilisées pour l'entraînement. Comme les données d'entraînement, ces données de validations doivent être représentatives et diversifiées, puisqu'elles permettent de vérifier les performances du modèles, et d'identifier un sur-ajustement si le modèle ne donne pas, après entraînement, des résultats satisfaisants sur cet ensemble. Les données de validation permettent de mettre en avant les pertes à mesure que le modèle s'ajuste aux données d'entraînement\footcite{carremansHandlingOverfittingDeep2019}.
	
	Pour prévenir le sur-ajustement d'un modèle\footnote{Il existe diverses méthodes pour corriger le sur-ajustement d'un modèle de détection après entraînement, cependant, ce mémoire traitant de l'intégration de la vision artificielle à une chaîne de traitement des sources par le prisme du lien avec les équipes de recherche en histoire, nous n'aborderons pas ces questions qui concernent plus spécifiquement les aspects techniques de la création d'un modèle de détection. \cite{carremansHandlingOverfittingDeep2019}}, il est nécessaire de prévoir un jeu de données d'entraînement au volume suffisant -- supérieur à plusieurs milliers d'exemples dans le cas d'images\footnote{Le projet \eida compte plusieurs dizaines de milliers de pages dans son jeu de données d'entraînement.} -- et à la diversité représentative des données que le modèle pourra rencontrer. Le modèle de détection entraîné présente une capacité de généralisation qui rend son application pertinente pour le traitement des sources : l'entraînement d'un modèle \textit{off-the-shelf} permet ainsi de partir d'un modèle relativement performant dans des contextes variés et d'obtenir un modèle répondant aux besoins du projet, en trouvant l'équilibre entre manque d'ajustement et sur-ajustement.
        \\
		
		Les modèles de vision \textit{off-the-shelf} ouvrent à des projets divers la possibilité d'intégrer la vision artificielle à leurs méthodologies, en réduisant le coût humain et temporel du développement d'un modèle de \textit{deep learning} par la mise à disposition en accès libre d'outils déjà performants, qu'il est possible d'entraîner pour les ajuster à des données spécifiques. Pour la détection d'objets, des jeux de données en accès libre tels qu'ImageNet permettent un pré-entraînement de ces modèles \textit{off-the-shelf}, qui apprennent alors des caractéristiques larges qui peuvent être précisées par un entraînement sur des sources plus spécifiques, en nécessitant un volume de données moins important qu'un modèle créé de zéro : ces modèles de détection \textit{off-the-shelf} présentent ainsi une solution aux limites que peuvent présenter les sources historiques en termes de volume des données disponibles, et permettent également aux projets de se construire sur des bases solides, sans allouer de ressources à la création d'outils déjà existants, déjà performants, pour des tâches telles que la détection d'objet qui font partie des tâches canoniques de la vision par ordinateur.
        \clearemptydoublepage
        
        \chapter[Construire une plateforme pour la détection]{Construire une plateforme pour la détection : outils, interfaces et modèles de données}
        
        L'intégration du \textit{deep learning} aux pratiques des chercheurs en sciences historiques passe par le développement d'une plateforme qui leur permet d'exploiter simplement ces outils pour traiter leurs sources. Cette plateforme à interface graphique doit être adaptée aux sources traitées, et pensée pour intégrer toutes les fonctionnalités souhaitées. En s'appuyant sur les développements réalisés par les projets \eida et \vhs, cette partie revient sur la construction d'une application  et d'une \api dédiées à la détection de diagrammes et images scientifiques dans les numérisations d'ouvrages manuscrits ou imprimés, avec une interface pour la correction de la détection : nous évoquons ainsi les méthodes de développement, les réflexions liées aux données, les besoins matériels liés à l'utilisation d'algorithmes de vision, et le développement d'une \api pour lancer l'inférence sur un \gpu. 
         
                \section{Penser une application pour la détection d'objets}
                    % Architecture de l’application : fonctionnalités et outils

\subsection{Une interface pour le traitement des sources}
    \subsubsection{Construire une application réutilisable}
    La création d'un modèle performant est une base essentielle à tout projet intégrant l'\ia à sa chaîne de traitement des sources : il s'agit du socle sur lequel repose l'efficacité des outils développés pour les chercheurs. Cependant, d'un point de vue pratique, les historiens n'interagissent pas directement avec ce modèle, il est donc nécessaire de développer une application qui sert de pont entre les utilisateurs et l'algorithme de détection, et qui permet ainsi le dépôt et le traitement des sources par l'intermédiaire d'une interface graphique, accessible et utilisable par les chercheurs\footnote{Les applications \eida et \vhs sont développées en prenant en considération leur utilisation future par un public plus large que les chercheurs affiliés à ces projets. Elles restent cependant axées, dans leur développement, à un public provenant du monde de la recherche, nous utilisons donc le terme \og chercheur \fg pour décrire ses utilisateurs. D'autres projets, tels que le projet \href{https://filigranes.hypotheses.org/}{Filigranes pour tous}, développent leurs outils autour des utilisateurs grand public ; nous les prenons en compte dans ce mémoire, en ayant conscience de la différence de réflexion que cela apporte sur la manière de construire une plateforme.}. Cette plateforme doit ainsi être développée en prenant en compte la diversité des sources étudiées par ses utilisateurs.
    
    La détection d'objet dans les images représente une première étape de nombreuses tâches de vision artificielle, et présente ainsi un intérêt pour de nombreux projets aux objectifs divers. Le développement d'une application pour la détection d'objet dans des images est donc d'intérêt pour plusieurs projets en cours, tels que \vhs et \eida : pour éviter à de multiples projets de développer des outils aux fonctionnalités similaires, ces deux projets ont fait le choix de construire une application réutilisable, dont le code sera proposé en accès libre sur GitHub\footnote{Le développement des applications \eida et \vhs étant toujours en cours, le code n'est à cette date par disponible en accès libre.}. Le développement des applications \eida et \vhs est collaboratif, porté par les ingénieurs des deux projets qui travaillent en parallèle pour la mise en place d'une plateforme adaptée aux besoins de chacun des projets : l'enjeu est donc de créer une application suffisamment spécifique pour répondre aux besoins des deux projets, tout en étant suffisamment généraliste pour être réemployée à l'avenir par d'autres projets.
    
    Il n'a, à ce jour, pas été développé d'interface pour les applications \eida et \vhs. Ces dernières sont accessibles pour les chercheurs qui demandent aux équipes d'ingénierie la création d'un identifiant donnant accès à l'interface administrateur des applications : celles-ci reposent sur l'interface par défaut proposée par Django (fig. \ref{fig:eida_admin}). Un développement spécifique des interfaces destinées aux utilisateurs\footnote{Le développement web frontal, ou \textit{front-end}, désigne les productions d'une application avec lesquelles l'utilisateur peut interagir directement.} est prévu, afin de rendre ces applications plus ergonomiques. L'objectif actuel étant de mettre à disposition des chercheurs partenaires des applications pour le dépôt et le traitement de leurs sources, celles-ci ne sont pas accessibles au grand public, et ont pour objectif de permettre la constitution de jeux de données d'entraînement par le biais d'une interface graphique. 
    
    \begin{figure}[h]
    	\centering
    	\includegraphics[width=16cm]{images/eida_admin.png}
    	\caption{Interface administrateur de l'application \eida}
    	\label{fig:eida_admin}
    \end{figure}
    
    \subsubsection{Développement en collaboration, programmation modulaire et remploi du code}
    Dans ce contexte, les ingénieurs des projets \eida et \vhs travaillent sur différentes branches d'un même dépôt GitHub, qui dispose alors d'une branche pour la mise en production de l'application \eida et d'une autre pour la mise en production de l'application \vhs. Ainsi, il est possible pour un projet d'exploiter les développements faits dans le cadre du second, ou de choisir de ne pas les employer s'ils ne semblent pas pertinents pour les besoins spécifiques du projet concerné. Ce développement collaboratif par le biais d'un dépôt GitHub assure une liaison constante entre les équipes des deux projets, tout en offrant la liberté de ne pas mettre en production tous les développements faits dans le cadre du projet jumeau.
    
    La programmation modulaire est une solution favorisée pour la création d'une application réutilisable, puisqu'elle permet le développement de modules indépendants qui répondent à des besoins spécifiques, et qui peuvent être réemployés par d'autres projets sans être dépendants du reste de l'application. Ainsi, les différents traitements appliqués aux numérisations déposées par les utilisateurs font appel à des outils divers, indépendants, dédiés chacun à une tâche spécifique du \textit{workflow} : cette disposition permet l'amélioration et la modification de chacun des modules sans impacter la structure globale de l'application, et permet également la récupération d'éléments spécifiques par des programmes futurs. Ainsi, le cœur de l'application \eida/\vhs permet le dépôt, le stockage et l'affichage des numérisations d'ouvrage, ainsi que la correction des annotations. La détection d'objet est gérée par une \api, dont le développement est détaillé dans la partie suivante.
    
\subsection{Décrire les sources : images et objets}
    \subsubsection{Modèle de données pour les ouvrages historiques}
    
    L'application \vhs est développée avec le \textit{framework} Django et adossée sur une base de données gérée avec PostgreSQL. Le modèle de données doit permettre une description des sources historiques des projets, tout en étant adapté aux besoins spécifiques liés aux outils de détection utilisés. Comme l'application, le modèle de données construit pour le projet a pour vocation d'être suffisamment spécifique pour répondre aux besoins de description des sources historiques de \vhs et \eida, tout en étant suffisamment généraliste pour être réemployé à l'avenir par des projets différents qui souhaiteraient appliquer des algorithmes de détection d'objet à des numérisations d'ouvrages.
    
    Le modèle de données initialement construit (fig. \ref{fig:vhs_data_model}) pour l'application \vhs prévoit l'existence de sources manuscrites et imprimées, qui correspond en effet aux support représentés dans les corpus d'\eida et de \vhs, mais présente rapidement des limites, notamment liées à la description de séries d'ouvrages contenant une même œuvre, ou à la description d'un ouvrage contenant plusieurs œuvres. De plus, la pertinence de la distinction absolue des manuscrits et imprimés dans la description des objets est remise en question, menant à la refonte de ce modèle de données pour une version basée sur le témoin comme unité centrale (Annexe \ref{eidaDataModels}). 
    
    Ce modèle, en cours d'implémentation dans l'application \eida, présente l'avantage d'une plus grande flexibilité, pertinente aussi bien pour les projets qui le construisent que pour des projets futurs qui souhaiteraient utiliser le code en accès libre de l'application développée. En l'absence d'objets spécifiques \textit{Manuscript} et \textit{Printed}, le modèle de données permet la description d'ouvrages aux configurations diverses, sans nécessiter dans l'architecture de l'application des chaînes de traitement indépendantes pour les manuscrits et les imprimés. Ce modèle est ainsi appuyé sur une entité générique représentant l’objet physique, le \textit{Witness}, tandis que le précédent modèle distinguait de manière plus nette les entités \textit{Volume}, \textit{Manuscript} et \textit{Printed}, résultant en un modèle de données moins centralisé autour de l'objet physique.
    
    L'entité \textit{Witness} désignant un objet physique permet l'introduction d'une granularité plus fine à travers les entités \textit{Content} et \textit{Work}, qui permettent la description de manuscrits composites contenant plusieurs œuvres : ce niveau de description est pensé pour faciliter, après l'implémentrecherchelgorithmes de détection de similarité, la comparaison des illustrations ou diagrammes dans différentes versions d'une même œuvre.
    
    L'entité numérisation (\textit{Digitization}) est centrale à ce modèle de données, en tant qu'élément de base d'une chaîne de traitement pour la détection d'objet : dans le modèle de données de la plateforme \vhs, la numérisation est rattachée au \textit{Volume} et non à l'entité \textit{Printed}, créant une distinction entre le traitement des sources manuscrites et imprimées. Le modèle de données modifié attache la \textit{Digitization} au \textit{Witness} pour dépasser cette différence de traitement, en considérant la numérisation d'un objet physique unique qu'est le témoin. Cette variation de l'entité \textit{Digitization} offre également la possibilité cruciale d'attacher plusieurs numérisations à un même \textit{Witness}, en liant l'identifiant du témoin à la numérisation et non la numérisation au manuscrit ou au volume, comme fait précédemment. La possibilité de lier plusieurs numérisations à un témoin permet, du point de vue de la détection d'objet, la comparaison des résultats de détection sur diverses versions numérisées, et de conserver les meilleurs résultats.
    
    Ainsi, le modèle de données est construit en prenant en compte, d'une part, les besoins des historiens pour la description des sources historiques, en mettant l'accent sur la résolution des différences matérielles entre manuscrit et imprimé pour obtenir un modèle global, et en projetant déjà, d'autre part, les traitements appliqués sur les numérisations des ouvrages avec des algorithmes de vision artificielle, de la détection d'objet à la recherche de similarité. Le modèle résultant de ces réflexions, menées en ateliers réunissant les équipes d'histoire des sciences et les équipes de vision par ordinateur, correspond ainsi aux besoins parfois divergents des différents acteurs des projets.

	\subsubsection{Images et objets détectés : une nouvelle source ?}
	
	Il est une entité que le modèle de données des projets \vhs et \eida ne prend à ce jour pas en compte : les objets détectés. La chaîne de traitement automatique n'étant pas déployée, les objets détectés ne comptent à ce jour pas parmi les éléments décrits ; il reste cependant nécessaire de considérer leur intégration au modèle de données dans le cadre d'une réflexion sur les étapes futures du développement, et sur les traitements qu'il est souhaitable d'appliquer aux sources historiques. L'algorithme de détection d'objet utilisé sur les numérisations de témoins génère en sortie une série d'images correspondant aux objets -- diagrammes ou illustrations scientifiques -- détectés dans les images des pages des ouvrages. Ces nouveaux objets image, stockés dans l'application, constituent une nouvelle entité non-exploitée à ce jour, mais qui présentera, dans le cadre des futures applications d'algorithmes de vectorisation ou de recherche de similarité, un intérêt particulier. Il est donc nécessaire d'envisager une extension du modèle de données pour l'intégration de ces objets, qui doivent être liés au témoin auquel ils appartiennent, en vue, par exemple, du développement de fonctionnalités d'édition ou pour permettre la comparaison des illustrations dans différentes versions d'une œuvre.
	
	Ces objets, exclus à ce jour du modèle de données décrit, représentent en effet l'élément d'intérêt des études qui pourront être faites des sources traitées. Les fonctionnalités envisagées, qu'il s'agisse de vectorisation ou de recherche de similarité dans les illustrations, ne s'appuient pas sur les numérisations complètes des témoins, mais exclusivement sur ces images extraites. Il est ainsi nécessaire que ces objets soient intégrés au modèle de données en tant qu'entité : leur exploitation requiert l'existence d'un lien avec le témoin -- ainsi qu'avec la numérisation -- duquel elles sont extraites, pour permettre leur identification. La détection constitue la première étape d'une chaîne de traitement plus longue\footnote{Les applications développées par \eida et \vhs étant à ce jour focalisées sur la création de données d'entraînement -- basées sur le modèle des jeux de données \yolov qui nécessite des images et les annotations au format texte -- et la détection d'objet, ces sorties ne sont pas actuellement exploitées. La chaîne de traitement faisant appel à \iiif pour l'affichage des images et de leurs annotations, aucun traitement n'est appliqué aux fichiers image des objets détectés, et le fichier texte suffit à l'indexation des annotations pour leur affichage dans un visualiseur \iiif. Nous décrivons l'architecture et la chaîne de traitement des sources dans les sections suivantes.}, et le modèle de données actuellement instauré peut être étendu dans le cadre de développements futurs ayant pour vocation la mise en place de ces fonctionnalités.

            
                \section{Annoter sur un GPU : extractorAPI}
                    % Créer un outil open source : penser une application réutilisable

\subsection{De VHS à EiDA : adapter un modèle de données}
    \subsubsection{Sous-sous-section 1}
    
    \subsubsection{Sous-sous-section 2}
    
    
    \subsection{Construire une application réutilisable}
        \subsubsection{Sous-sous-section 1}


        \subsubsection{Sous-sous-section 2}


        \\
        
        L'\textit{open source}\footnote{Le terme d'\textit{open source}, ou code source ouvert, désigne la pratique de mise à disposition du code source d'un logiciel sous licence libre, pour permettre gratuitement sa réutilisation, sa distribution et sa modification.} est une préoccupation centrale lors du développement d'outil pour les projets de recherche : il est en effet souhaitable de produire, dans la mesure du possible, des applications réutilisables par des projets futurs, pour assurer une continuité dans les travaux produits sans que se démultiplient les développements d'outils similaires. Ainsi, le remploi du code est au cœur des ambitions des équipes d'ingénierie : l'intégration de l'apprentissage profond à des projets d'histoire nécessite des outils spécifiques, construits autour de ces traitements des sources par l'\ia, qui restent néanmoins similaires -- ou comparables dans leurs besoins techniques -- d'un projet à l'autre. L'utilisation d'un \gpu pour la vision artificielle compte parmi les besoins matériels qui se traduisent entre les projets, ainsi, dans une optique de programmation modulaire, une \api externe à l'application \vhs/\eida a été développée pour répondre à ce besoin, sans être hautement spécifique aux besoins de ces projets comme pourrait l'être un module intégré à l'application. \exapi, l'\api développée dans ce cadre, est ainsi conçue, dès son cahier des charges, pour répondre aux besoins du projet \eida tout en restant réemployable dans d'autres contextes -- sont ainsi utilisés, pour sa construction, des outils libres qui en standardisent le développement, avec la vocation de produire une \api aussi flexible que robuste.
        \clearemptydoublepage
            
      	\chapter[\textit{Computer vision} et pratiques des chercheurs]{Intégrer la vision artificielle aux pratiques des chercheurs}
      	
      	Le chapitre précédent établit, du point de vue du \textit{back end}, les besoins techniques pour l'intégration de la vision artificielle à une chaîne de traitement des sources historiques : ces préoccupations en termes d'architecture et de puissance de calcul sont celles des équipes d'ingénierie, et existent en parallèle des préoccupations qui concernent plus directement les utilisateurs, et impactent réellement les pratiques des chercheurs. Ce chapitre revient ainsi sur la chaîne de traitement des sources historiques dans sa totalité, depuis la numérisation fournie par l'utilisateur jusqu'aux résultats retournés pour le traitement par les chercheurs, et sur les échanges avec les utilisateurs, en termes d'interface mais également en termes de médiation.
               
                \section[\textit{Workflow} de traitement des sources]{De la numérisation à l’annotation : l'automatisation du traitement des sources}
                    % De la numérisation à l’annotation

\subsection{Chaîne de traitement des sources historiques}
    \subsubsection{Du manuscrit aux diagrammes : chaîne de traitement des sources astronomiques}
	La chaîne de traitement proposée par \eida (fig. \ref{fig:eida_workflow}) propose ainsi une alternance d'étapes automatisées, en bleu sur le schéma, et d'étapes d'analyse par les chercheurs du projet.
	
	\begin{figure}[h]
		\centering
		\includegraphics[width=15cm]{images/eida_workflow.png}
		\caption{\textit{Workflow} de traitement des sources du projet \eida}
		\label{fig:eida_workflow}
	\end{figure}

    \subsubsection{Traitements automatiques, traitements manuels}
	qu'est-ce qui peut être automatisé, qu'est-ce qui ne l'est pas, intervention des chercheurs, etc etc
    
    \subsection{Échanges et transformation des données}
        \subsubsection{Sources d'entrée, sources de sortie}
		de l'image au manifeste à l'image, au txt puis sas

        \subsubsection{Traitement des résultats de la détection}
        après avoir obtenu les diagrammes : deux possibilités, plus de traitements automatiques, cf partie 3, ou résultat en lui même
        
        par exemple détection d'objets permet de constituer un corpus qui sera traité manuellement ?

    \subsection{Créer des interfaces pour les étapes manuelles}
        \subsubsection{Correction par les chercheurs : au-delà de l'entraînement}
        expliquer que la correction se fait pas juste pour l'entraînement mais que l'interface permettra plus tard aux chercheurs de sélectionner les diagrammes qui les intéressent

        \subsubsection{Interface d'annotation}
		interface pour les chercheurs, pour la correction : captures d'écran 
		réelles réflexions techniques pour la praticité 

             
                \section{Médiation et documentation}
                    % Médiation et documentation

\subsection{Échanges avec les chercheurs}
ANNEXE PPT SÉMINAIRE DH
    \subsubsection{Séminaires DH}

    
    \subsubsection{Ateliers d'annotation}

    
    \subsection{Documentation et Wiki : penser le réemploi du code}
    ANNEXE PAGE WIKI
        \subsubsection{Commentaires dans le code}


        \subsubsection{Rédaction du Wiki}


        \\
        
        L'intégration de la vision artificielle aux pratiques des chercheurs en humanités passe ainsi par l'établissement d'une chaîne de traitement qui trouve un équilibre entre besoins scientifiques et possibilités numériques, en mêlant étapes automatiques et correction manuelle pour s'assurer de la pertinence des données qui en découlent. L'automatisation n'est jamais totale : le traitement de sources historiques ne peut s'effectuer sans un regard critique, humain, qui contrebalance les limites de l'\ia qui ne permettent pas d'appréhender les objets étudiés sous tous les aspects nécessaires à prendre en compte dans le cadre d'une recherche historique. L'apprentissage profond et la vision artificielle s'insèrent donc dans les pratiques de chercheurs en histoire par le biais d'outils et d'interfaces qui appellent à leur contribution, à leurs corrections, et à leur regard sur les résultats produits par des algorithmes : les ingénieurs ont pour rôle, dans ce contexte, de développer des outils et pratiques qui permettent ce lien entre les méthodes et techniques du \dl et les chercheurs en humanités, et d'établir une médiation fluide qui assure la pertinence du travail produit. Au-delà de la communication avec les chercheurs, utilisateurs des interfaces, il est du devoir des ingénieurs d'assurer la pérennité des outils créés par une documentation destinée à d'autres développeurs, pour permettre la maintenance du code, ou son réemploi dans des projets extérieurs : ces bonnes pratiques assurent ainsi la vie des développements produits, et leur bonne intégration dans un contexte de recherche plus large, en tant qu'outils libres pour le traitement des sources.
        \clearemptydoublepage

    \part{Perspectives pour le traitement des sources : vers un outil pour l’édition et la recherche}
        \chapter[Éditer des diagrammes]{Éditer des diagrammes : vectorisation et édition critique}
        La partie précédente s'attache à décrire l'intégration des techniques et méthodes de la vision artificielle aux pratiques des chercheurs, en s'appuyant notamment sur le développement d'outils pour l'utilisation d'algorithmes de détection d'objets. Tâche canonique de la vision artificielle, la détection d'objets dans les images est une pratique établie, dont l'intérêt pour la navigation des corpus est avéré par un nombre certain de projets récents, tels que CorDeep\footcite{CorDeep}, ayant produit et publié des applications intégrant des algorithmes de détection d'objets pour le traitement des sources.
        
        La vision artificielle est riche en possibilités : au-delà des outils de détection plus accessibles -- dont des modèles \textit{off-the-shelf} existent pour les projets qui souhaiteraient l'employer -- les traitements automatiques pour l'édition ou pour l'étude critique des sources sont considérés et étudiés. La détection d'objet n'est ainsi pas considérée comme une finalité, mais comme une première étape dans une chaîne de traitement qui intègre des étapes automatique de détection de similarités, de \textit{clustering} ou de vectorisation\footnote{Par vectorisation, nous entendons la transcription de numérisations de diagrammes géométriques en images \svg.}, offrant de nouvelles perspectives dans le cadre de l'intégration de la vision artificielle à l'étude des sources. Pour des projets portés sur l'illustration scientifique, tels que le projet \eida, la vectorisation présente un intérêt particulier, ouvrant des possibilités en termes d'édition automatique des diagrammes.
        
                \section{Édition numérique des diagrammes astronomiques}
                    % Édition critique des diagrammes astronomiques

\subsection{Historique}
    \subsubsection{Sous-sous-section 1}

    
    \subsubsection{Sous-sous-section 2}

    
    \subsection{Éditer les diagrammes en format numérique}
        \subsubsection{Sous-sous-section 1}


        \subsubsection{Sous-sous-section 2}


            
                \section[De l’image aux vecteurs]{\label{vectorEdition}De l’image aux vecteurs : la vision artificielle pour l’édition numérique}
                    % De l’image aux vecteurs : la vision artificielle pour l’édition numérique

\subsection{Automatiser la transcription}
	\subsubsection{Diagrammes en SVG}
	Le \svg est un format basé sur \acrshort{xml} permettant de décrire trois types d'objets graphiques en deux dimensions : les formes vectorielles\footnote{Pour la description de diagrammes astronomiques, constitués de droites, de courbes, de cercles, les formes vectorielles sont particulièrement appropriées.}, les images et le texte. Contrairement aux autres formats d'image -- PNG et JPEG -- mentionnés dans ce mémoire, le SVG n'est pas basé sur des pixels, mais sur des formes géométriques. Les coordonnées et la structure des objets vectoriels sont décrits dans une document \acrshort{xml} : il s'agit d'un format ouvert, standard, dont l'intégration sur le web est aisée. Le \svg est doté de nombreuses fonctionnalités pour la description d'objets complexes, dynamiques, et possède de nombreux avantages face aux formats d'image basés sur des pixels.
	
	La possibilité de l'agrandissement ou de la réduction de la taille de l'image sans perte de résolution est le principal avantage des images vectorielles en \svg\footnote{Le format \svg présente des limites en termes de description d'images complexes ; cependant, ce mémoire traitant essentiellement de diagrammes géométriques aux formes simples, nous ne rencontrons pas cet inconvénient, et ne le détaillons pas.}. Il s'agit souvent de fichiers moins volumineux que les images en pixels, et qui permettent de traiter le texte contenu dans l'image comme tel, le rendant ainsi lisible. Ainsi, le format \svg est, pour l'étude et l'analyse de diagrammes, un format plus manipulable, qui permet d'envisager la superposition de plusieurs diagrammes pour leur comparaison, de modifier la taille des images pour créer des éditions numériques, ou d'intégrer des légendes et métadonnées qu'il sera possible de rechercher pour faciliter la navigation d'un corpus d'images de diagrammes et trouver, par exemple, les diagrammes figurants des éléments similaires.
	
	Pour l'intégration de toutes ces fonctionnalités, la vectorisation des images de diagramme est un des objectifs du projet \eida, qui vise à mettre en ligne une plateforme qui permettra aux utilisateurs de traiter leurs sources et d'obtenir des images \svg sans employer de logiciels spécialisés.
	
	\subsubsection{\textit{Pipeline} de vectorisation automatique}
	La vectorisation automatique de diagrammes à partir de sources historiques s'appuie sur un ensemble de tâches de vision artificielle, qu'il est possible de décomposer en trois tâches principales. À partir d'une numérisation de diagramme, un premier algorithme de détection de texte est appliqué, pour supprimer les éléments textuels qui pourraient parasiter les tâches de détection suivante, appliquées spécifiquement à l'image et ne devant donc pas prendre en compte les potentielles étiquettes ou légendes qui peuvent accompagner un diagramme scientifique. Après exclusion du texte, un algorithme de détection de contours, ou \textit{edge detection}\footnote{La détection de contours est une tâche bien établie de la vision artificielle, qui s'appuie sur la détection de changements d'intensité lumineuse pour détecter la structure des objets, et des algorithmes tels que celui de Canny, employé dans le modèle développé pour le projet \eida, permettent la création de modèles performants en termes de détection et de clarté des résultats. \cite{cannyComputationalApproachEdge1986}}, est appliqué pour la détection des lignes du diagramme, et la production d'une image binaire des contours du diagramme. Cette image est ensuite traitée par un algorithme de détection des lignes et des cercles, basé sur l'algorithme \acrfull{ransac}\footnote{L'algorithme \acrshort{ransac} est une méthode itérative pour l'estimation de paramètres de modèles mathématiques basée sur un modèle contenant des \textit{inliers} (données pertinentes correspondant au modèle) et \textit{outliers} (données aberrantes). L'algorithme sélectionne aléatoirement un ensemble de points dans l'ensemble de données, qui sont utilisés pour estimer un modèle. À partir de ce modèle, l'algorithme détermine les points de l'ensemble complet qui sont conformes à ce modèle. Par répétition de ces étapes, l'algorithme sélectionne le modèle avec le plus grand nombre de points pertinents, et le sélectionne comme modèle final. \cite{derpanisOverviewRANSACAlgorithm2010a}}, qui produit en sortie une prédiction correspondant à une transcription en \svg du diagramme\footnote{Si l'emploi unique de lignes et de cercles peut sembler limité, il est suffisant dans le cas de diagrammes géométriques, et permet une transcription presque totale de ces illustrations.}. 
	
	Cette \textit{pipeline} automatique a fait l'objet de tests sur les données du projet \eida, et a retourné des résultats satisfaisants sur un jeu de données de quatorze diagrammes astronomiques (fig. \ref{fig:imagine_vector}). Les chercheurs du laboratoire \imagine, chargés du développement de la méthode de vectorisation, reprochent cependant à la méthode décrite ses faibles capacités de généralisation\footnote{La généralisation définit la capacité à produire de bonnes prédictions sur de nouvelles sources.}, et proposent, dans les prochaines étapes de son développement, une approche par le \dl\footnote{La méthode décrite est le résultat des travaux de recherche de master de Syrine Kalleli dans le cadre de son stage dans le laboratoire \imagine, sous la supervision de Mathieu Aubry.}.
	
	\begin{figure}[h]
		\hspace{1pt}
		\begin{subfigure}{1\linewidth}
			\centering
			\includegraphics[width=16cm]{images/imagine_vector1.png}
		\end{subfigure}
		\hspace{1pt}
		\begin{subfigure}{1\linewidth}
			\centering
			\includegraphics[width=16cm]{images/imagine_vector2.png}
		\end{subfigure}
		\caption{Résultats préliminaires de la \textit{pipeline} automatique de vectorisation sur les données du projet \eida}
		\label{fig:imagine_vector}
	\end{figure}

	\subsubsection{Intégrer la vectorisation à une chaîne de traitement automatique}
	La \textit{pipeline} automatique de vectorisation décrite peut exister dans le contexte d'une chaîne de traitement plus vaste, liée à d'autres tâches automatiques des sources pour minimiser les traitements manuels des chercheurs sur leurs documents d'étude. 
	
	Dans une chaîne de traitement telle que celle proposé par le projet \eida, la vectorisation des diagrammes intervient à la suite d'une détection automatique des objets dans les numérisations, et de la correction par les chercheurs de ce premier traitement. Ainsi, les algorithmes pour la vectorisation appliquent des traitements à des images de diagrammes qui ne comprennent plus l'ensemble d'une page numérisée, mais qui ne correspondent qu'aux limites de l'illustration détectée par le précédent algorithme. Ce premier traitement par un algorithme de détection d'objet assure la performance des algorithmes de vectorisation, tout en automatisant l'étape d'extraction des diagrammes. Entre ces deux étapes automatiques, il est préférable d'inclure une intervention manuelle, qui permet au chercheur de sélectionner les illustrations d'intérêt parmi les éléments détectés par le premier modèle, pour éviter ainsi la perte de temps que constituerait la vectorisation d'illustrations qui n'entrent pas dans le cadre du projet de recherche. 
	
	L'intégration à une plateforme d'un algorithme de détection permettant l'extraction des illustrations préalablement à la vectorisation de ces images de diagramme permet une intervention minimale du chercheur, qui sélectionne les numérisations en amont de la chaîne de traitement. Le chercheur intervient ensuite pour la correction de la détection : ainsi, la fouille dans le corpus est automatisée, permettant le traitement rapide d'un grand volume de numérisations, et la transformation des illustrations détectées puis choisies en un objet vectoriel aisément exploitable est effectuée par une suite d'algorithmes qui ne nécessite pas, pour le chercheur, de compétences techniques particulières. Cette succession permet de s'assurer du lancement de la vectorisation sur un ensemble d'images pertinent, et leur traitement automatique pour l'étape chronophage qu'est la transcription. À l'issue de ce processus, les transcriptions automatiques -- comme tout traitement par un modèle de vision artificielle -- doivent faire l'objet d'une correction, pour s'assurer de la justesse des prédictions et de la pertinence des données exploitées ou publiées.

\subsection{Données d’entraînement pour la vectorisation et correction des annotations}
    \subsubsection{Jeux de données et pratiques d'annotation}
    Pour l'entraînement et la validation d'une \textit{pipeline} de vectorisation, il est nécessaire de produire une vérité de terrain à partir d'annotations corrigées, élément indissociable de la création d'outils employant des algorithmes d'\ia. Les formats de ces jeux de données correspondent aux formats d'entrée et de sortie des algorithmes utilisés : pour la vectorisation, les données d'entrée sont des images au format PNG ou JPEG, et les prédictions sont des images vectorielles en \svg. La vérité de terrain se doit donc d'être constituée d'un ensemble de numérisations de diagrammes accompagnées d'annotations, c'est-à-dire de leur transcription en images vectorielles (fig \ref{fig:diagram_svg}).
    
    \begin{figure}[h]
    	\begin{subfigure}{0.5\linewidth}
    		\centering
    		\includegraphics[width=8cm]{images/diagram8.png}
    	\end{subfigure}
    	\hspace{1pt}
    	\begin{subfigure}{0.5\linewidth}
    		\centering
    		\includegraphics[width=8cm]{images/diagram8_corr.png}
    	\end{subfigure}
    	\hspace{1pt}
    	\caption{Numérisation d'un diagramme astronomique et capture d'écran de sa transcription corrigée}
    	\label{fig:diagram_svg}
    \end{figure}

	Pour la création d'une vérité de terrain, il est ainsi nécessaire d'établir des pratiques uniformes entre tous les membres d'un projet, pour s'assurer de la pertinence des résultats obtenus, basés sur ces données. Le projet \eida utilise pour la correction de ces annotations le logiciel Inkscape\footnote{Inkscape est un logiciel destiné à la création d'images vectorielles. L'utilisation du logiciel dans le projet \eida a fait l'objet d'une session de formation informelle pour la prise en main de l'outil. \cite{Inkscape}}, qui permet l'import des résultats primaires de la vectorisation pour sa correction par les chercheurs. 
	
	La méthode de vectorisation ne produisant les objets vectoriels de sortie qu'avec des cercles et des lignes, l'annotation a ainsi fait l'objet de discussions pour établir, en fonction de ces limites de la technique, les décisions scientifiques les plus pertinentes. À titre d'exemples, il est décidé que les cercles dont le tracé manuscrit est hésitant dans la source ne font l'objet que d'un unique cercle dans la version vectorielle. De même, la méthode de vectorisation ne prend pas en compte les arcs de cercle, qu'elle transcrit comme des cercles complets. Il a été décidé, après discussion, que ces arcs doivent être annotés comme des courbes et non des cercles complets, dans l'espoir d'entraîner le modèle à ce tracé spécifique. Comme pour la détection, les sources présentent de nombreux cas limites : les manuscrits d'astronomie peuvent contenir, notamment, des dessins d'astrolabe, qui dans leur forme présentent des similarités avec les diagrammes et sont d'intérêt d'un point de vue scientifique, mais qui, de par leur complexité, ne peuvent être transcrits par le modèle. Les ateliers d'annotation menés permettent ainsi d'établir un pont entre les attentes scientifiques et les possibilités techniques, prometteuses mais encore limitées, des méthodes de vectorisation automatique\footnote{Pour assurer l'uniformité des données d'entraînement d'\eida, un document de spécifications a été rédigé par les ingénieurs du projet.}. En considérant qu'un modèle de vision artificielle, même entraîné, n'est jamais aussi précis qu'un annotateur humain, et que les transcriptions comporteront toujours des erreurs, les décisions sont prises pour l'obtention d'un modèle de vectorisation maximaliste, qui génère des faux positifs plutôt que des faux négatifs\footnote{Cette décision découle d'un constat pratique : lors de la correction des annotations, il est plus aisé de supprimer des éléments que d'en ajouter. Ce constat est valable pour la vectorisation, mais aussi pour la détection d'objets, ou pour les tâches décrites dans le chapitre suivant.}.

    \subsubsection{Une interface pour l'annotation ?}
    Un projet souhaitant avoir recours à la vision artificielle pour la vectorisation automatique d'illustrations se doit de prendre en compte la nécessité d'une interface pour la correction des annotations. Pour assurer la production de données fiables et pertinentes, une intervention manuelle par un annotateur capable de porter un regard critique sur les résultats produits est nécessaire. Dans le cadre du développement d'une plateforme à destination des chercheurs, qui intègre de tels algorithmes, il est donc nécessaire de concevoir une interface destinée à cette correction, qui permet d'intégrer cette étape à une chaîne de traitement globale, et d'éviter aux chercheurs le recours à un logiciel tiers. En prenant en considération ces besoins, \eida prévoit le développement d'une interface intégrant l'application GeoGebra pour le traitement manuel des vectorisations\footnote{Cette fonctionnalité constitue une étape future du développement de l'application \eida, le cahier des charges de l'interface n'a, à ce jour, pas été établi. Les solutions proposées sont donc abstraites ; il a cependant été constaté que GeoGebra est un outil relativement connu par les chercheurs du projet.}. Cette interface future, intégrant le modèle développé par le laboratoire \imagine, vise ainsi à accélérer le travail d'édition des diagrammes astronomiques en minimisant le temps dédié à la transcription, et à proposer un outil libre, à la prise en main aisée, à destination des chercheurs.
        \\
        
        La vectorisation automatique de diagrammes géométriques ouvre de nombreuses perspectives en termes de soutien à la recherche, et permet d'envisager des outils et interfaces à destination des chercheurs qui proposeraient d'automatiser des tâches autrement chronophage. La transcription des diagrammes, en lien avec l'histoire récente de l'édition critique de ce type d'illustration, fait l'objet d'un nombre restreint -- si ce n'est inexistant -- de normes qui en encadrent la pratique : les méthodes appliquées pour la transcription et l'édition numérique sont donc chronophages, variées, spécifiques aux projets qui les emploient, et les objets informatiques générés ne sont pas nécessairement manipulables ou interopérables. Le \svg, format libre, est particulièrement appropriés pour l'étude des diagrammes : la transformation d'une numérisation en objet vectoriel devient alors une étape clé de la chaîne de traitement. Manuellement, cette dernière est chronophage, et demande la connaissance d'outils spécifiques pour cette pratique. Ainsi, l'intégration à cette chaîne de traitement d'algorithmes de \cv permet de l'envisager comme une pratique systématique, rendant possible la transcription d'un volume important d'images en un temps moindre. La création d'outils libres, accessibles et aisés à prendre en main intégrant ces méthodes a vocation à faciliter cette étape chronophage du processus d'édition, et de permettre une analyse des diagrammes tirant réellement parti des avantages du numérique.    
        \clearemptydoublepage
        
        \chapter{Détection de similarité et \textit{clustering}}
	             \section{Trouver des motifs dans les données : le \textit{clustering}}
        			% Similarity retrieval et navigation des corpus

\subsection{Principe et méthodes}
    \subsubsection{Sous-sous-section 1}

    
    \subsubsection{Sous-sous-section 2}


    \subsection{Fouiller des corpus par l’image}
        \subsubsection{Sous-sous-section 1}


        \subsubsection{Sous-sous-section 2}


        			
                \section{\textit{Similarity retrieval} et étude de la circulation des idées}
                    % Similarity retrieval et analyse des corpus

\subsection{Principe et méthodes}
    \subsubsection{Sous-sous-section 1}

    
    \subsubsection{Sous-sous-section 2}


    \subsection{La vision artificielle, nouveau regard sur les sources ?}
        \subsubsection{Sous-sous-section 1}


        \subsubsection{Sous-sous-section 2}
        
        BIEN INSISTER ICI SUR LES LIMITES ET SUR LES CRITIQUES DES HISTORIENS DE L'ART


            
        \clearemptydoublepage
    
    \chapterNo{Conclusion}
    Étudier et exploiter les résultats automatiques : limites et perspectives pour les sciences historiques
    \addcontentsline{toc}{chapter}{Conclusion}
    \clearemptydoublepage
    
\appendix
    \part*{Annexes}	
    \addcontentsline{toc}{part}{Annexes}
    
    \chapter[Prepare custom data for training]{\label{YOLOv5Training}Prepare custom data for training using the YOLOv5 workflow}
	    \section{What is \yolov?}
	\begin{itemize}
		\item Built to detect objects in images
		\item Trained with a dataset of real images (ImageNet)
		\item Default model based on 80 object classes
	\end{itemize}

The goal of the training is to \textbf{adapt the model} to \textbf{our images and our classes}, for it to detect the objects we want it to.

\section{Training dataset}
	\begin{itemize}
		\item \textbf{Annotate the objects} of interest on the images to \textbf{create labels} : the objects are defined by the \textbf{position and dimensions of the rectangular box} surrounding them;
		\item The label file is a \texttt{.txt} file;
		\item In the label file, each object corresponds to \textbf{one row}: \texttt{class x\_center y\_center width height}
		\item Organize labels and images in directories: \textbf{one directory for images and one for labels}, each containing a \textbf{training, validation and testing dataset};
		\item \textbf{Training is 80\% of the dataset}, validation is either 20\% or 10\% if a testing dataset (10\%) is needed;
		\item The training and validation datasets must be \textbf{different data};
		\item The image file and labels file must \textbf{have the same name}.
	\end{itemize}

\section{Config file}
The \texttt{dataset.yaml} is the \textbf{configuration file for the training}. It contains:
	\begin{itemize}
		\item The \textbf{dataset directory path};
		\item The relative paths to the training, validation and test directories;
		\item A \textbf{class names dictionary}: in our case, the only class is \texttt{0: illustration}.
	\end{itemize}


\begin{verbatim}
	path: ../dataset # dataset root dir
	train: images/train  # training images (relative to 'path')
	val: images/val  # validation images (relative to 'path')
	test:  # test images (optional)
	
	# Classes names:
	0: illustration
\end{verbatim}

\section{Notes}
	\begin{itemize}
		\item The training dataset is made of the \textbf{images the researchers annotate};
		\item Our model needs to be able to detect \textbf{each illustration individually}rather than extracting illustrated pages as a unique illustration;
		\item Our model needs to be able to differentiate tables from diagrams;
		\item The model is trained on both images of \textbf{manuscripts and printed books}: does it impact the quality of the extraction?
	\end{itemize}

	    \clearemptydoublepage
    \chapter[Modèles de données EIDA/VHS]{\label{eidaDataModels}Modèles de données des applications VHS et EIDA}
	    \section{Modèle de données initial de l'application VHS}
	\begin{figure}[H]
		\centering
		\includegraphics[width=16cm]{images/vhs_data_model.png}
		\caption{Modèle de données de l'application \vhs avant refonte}
		\label{fig:vhs_data_model}
	\end{figure}

\section{Modèle de données de l'application VHS/EIDA}
	\begin{figure}[H]
		\centering
		\includegraphics[width=16cm]{images/eida_data_model.png}
		\caption{Nouveau modèle de données de l'application \eida appliqué à \vhs}
		\label{fig:eida_data_model}
	\end{figure}
     	\clearemptydoublepage
    \chapter{\label{exapiCahier}extractorAPI : cahier des charges}
	    \section{Fonctionnalités}
L’\api permet de déplacer l’annotation des images envoyées par les utilisateur-rice-s de \eida sur le \gpu : dans une logique de développement modulaire, le modèle entraîné n’est pas intégré directement à l’application, mais dans une \api indépendante. Cette séparation du corps de l’application et du module générant les annotations permet d’éviter le ralentissement général de l’exécution de l’application sur le serveur Web : l’utilisation d’une \api permet de profiter de la capacité de calcul du \gpu lors de l’inférence, alors que l’application est développée sur un serveur différent. La communication en \ssh entre le \gpu et l’application \eida n’étant pas satisfaisante du point de vue de la sécurité, le développement d’une \api permet de gérer ces échanges par une voie sécurisée.

Du point de vue de l’open source, cette forme de développement permet la création d’un outil plus flexible, qui pourra être réutilisé par d’autres projets en fonction de leurs besoins et de leur propre architecture.

L’\api reçoit une requête \http de la part de l’application \eida lorsque l’utilisateur-rice envoie une nouvelle numérisation de son témoin dans la base de données via un formulaire. Le manifeste généré par l’application est envoyé à un \api endpoint qui l’enregistre et en extrait les images, qui sont ensuite traitées par le modèle. Les annotations générées sont retournées à l’application en réponse à la requête, et peuvent ainsi être visualisées par l’utilisateur-rice.

L’objectif d’extractor\api – par rapport à notre \textit{workflow} actuel – est d’automatiser la détection des diagrammes par le modèle, pour que celle-ci puisse être faite à la demande de l’utilisateur-rice, sans intervention humaine, et que les annotations lui soient retournées automatiquement après envoi des numérisations.

	\subsection{\textit{Workflow}}
		\begin{enumerate}
		\item Envoi d’une numérisation par l’utilisateur-rice sur l’application \eida ;
		\item Création automatique d’un manifeste \iiif à partir des images ou du manifeste ;
		\item Envoi du manifeste \iiif via requête \http POST à l’\api endpoint ;
		\item Appel de la fonction Python : 
		\item Vérification et enregistrement du manifeste ;
		\item Enregistrement des images issues du manifeste pour leur traitement (\iiif downloader) ;
		\item Traitement des images par le modèle : création des annotations (fichiers .txt) ;
		\item \textit{Return} : envoi des annotations à l’application ;
		\item Affichage des images annotées à l’utilisateur-rice.
		\end{enumerate}
	
	L’architecture actuelle de l’application implique d’enregistrer deux fois les images : une première fois dans l’application \eida pour générer le manifeste \iiif,  puis une seconde fois sur le \gpu pour lancer l’inférence. On peut envisager le déplacement du serveur Cantaloupe sur le \gpu pour éviter la duplication de cette tâche.

	\subsection{Outils}
	L’\api est développée avec Flask, plus simple et flexible que Django pour la construction d’une \api légère, et le modèle est déployé à l’aide du module PyTorch.

		\subsubsection{Modèle}
		Le modèle utilisé a pour base docExtractor – appuyé sur le réseau de neurones U-Net –, un modèle entraîné sur des images synthétiques de manuscrits générées par l’algorithme SynDoc et sur ImageNet. Le script d’inférence utilisé est celui de YOLOv5.
		Il est nécessaire de prendre en compte, dans le développement de l’\api, que d’autres modèles pour la détection d’objets seront testés, et que sera conservé le modèle produisant les meilleurs résultats. 
		Le modèle final sera un modèle entraîné, idéalement commun avec \vhs. Il est cependant envisageable d’avoir un modèle par type de source (manuscrit ou imprimé), ce qui nécessiterait d'adapter la structure de l'\api pour pouvoir choisir le modèle utilisé en fonction de la requête.
		
		\subsubsection{\textit{Task queues}}
		Pour éviter la surcharge de l’\api et gérer la réception de multiples requêtes, il est nécessaire de mettre en place un task manager comme Celery, qui traite en tâche de fond les requêtes reçues, et permet d’imposer qu’elles soient traitées successivement. 
		
	\subsection{Échange de données}
		\subsubsection{Données envoyées}
		Le manifeste \iiif généré par \eida. 
		S’il est décidé après le réentraînement de conserver un modèle pour chaque type de témoin (manuscrit ou imprimé), il est nécessaire d’envoyer une donnée sur le type de support pour choisir le modèle correspondant.
		\subsubsection{Données reçues}
		Un fichier d’annotations au format \texttt{.txt} par manifeste. Celui-ci contient, pour chaque image :
		\begin{itemize}
			\item le numéro de l’image généré par une fonction \texttt{enumerate()},
			\item le nom du fichier image annoté, 
			\item les coordonnées de la détection au format \texttt{x y width height}.
		\end{itemize}

		Par exemple, pour deux objets détectés sur la 21e image du manuscrit 22 :
		\begin{lstlisting}
			21 ms22_0021.jpg
			514 907 1700 1685
			2424 831 1441 1423\end{lstlisting}

	\subsection{Sécurité}
	Pour assurer la sécurité de l’\api, une authentification avec un token a été envisagée. Pour plus de simplicité, l’\api utilise un décorateur pour \textbf{restreindre les hôtes autorisés} à envoyer des requêtes aux \textit{endpoints} : seuls l’hôte \texttt{obspm.fr} peut envoyer des requêtes pour l’inférence, évitant ainsi un risque de surcharge par des requêtes envoyées depuis un autre hôte.
	
	\subsection{Interactions}
	L’annotation des images est automatique après envoi de la numérisation ; il est cependant nécessaire de prendre en compte le temps de réponse de la requête du point de vue de l’utilisateur-rice.
	 
	L’application \eida répond actuellement à cette problématique en affichant le message flash suivant : “Le processus de conversion de.s fichier.s PDF en images et/ou d'extraction des images à partir de manifeste.s externe.s est en cours. Veuillez patienter quelques instants pour corriger les annotations automatiques.”
	
	On peut envisager de désactiver les boutons de visualisation et d’édition du manifeste jusqu’à réception de la réponse à la requête.

	    \clearemptydoublepage
	\chapter{\label{yoloScript}Script YOLOv5 pour lancer la détection d'objet}
		Le script suivant est une reproduction du script \texttt{detect\_vhs.py} utilisé dans \exapi pour lancer la détection d'objet dans les images envoyées par l'application : \url{https://github.com/jnorindr/extractorAPI/blob/main/yolov5/detect_vhs.py}
		\input{templates/annexes/yoloscript}
	    \clearemptydoublepage
	\chapter{\label{DHSeminar}DH Seminar: extractorAPI}
		La présentation suivante constitue le support d'un séminaire d'humanités numériques donné le 13 juin 2023 en présence de l'équipe du projet \eida. Le séminaire visait à présenter l'\api développée dans le cadre du stage.
		\includepdf[nup=1x2,pages=-, scale=0.8, delta=0 10mm]{templates/annexes/dhseminar.pdf}
	    \clearemptydoublepage
	\chapter[Annotate images with extractorAPI]{\label{exapiAnno}Annotate images from EiDA with extractorAPI}
		Ce document est extrait du Wiki de l'\api, disponible sur GitHub en accompagnement du code. Il a été rédigé pour documenter les interactions entre \exapi et l'application \eida, pour faciliter la réutilisation de l'\api par des projets tiers. Le document est disponible à l'adresse suivante : \url{https://github.com/jnorindr/extractorAPI/wiki/Annotate-images-from-EiDA-with-extractorAPI}
		\section{Workflow}
	\begin{enumerate}
	\item User submits a witness
	\item Verification: did the user send a digitization?
	\item A flash message informs the user of the delay for receiving the annotations
	\item Images are downloaded and \iiif manifest is generated
	\item POST request is sent from \eida to \exapi with manifest link and type of witness as parameters
	\item \api: images are downloaded, detection is launched and annotation file is generated
	\item POST request is sent from \exapi to \eida with annotation file
	\item Annotations are indexed to be displayed for the user
	\end{enumerate}

\section{Settings}
	\subsection{EiDA}
	In the \texttt{.env} file, the \texttt{EXAPI} variable must be set with the \URL of the \api.
	\begin{lstlisting}
		EXAPI="<gpu-api-address>"\end{lstlisting}
	
	In the settings file \texttt{settings.py}, complete the \texttt{GPU\_PORT} variable so the \texttt{GPU\_URL} variable can be set.
	\begin{lstlisting}
		GPU_PORT = 5000
		GPU_URL = f"{ENV('EXAPI')}:{GPU_PORT}"\end{lstlisting}

	\subsection{extractorAPI}
	In the \texttt{.env} file, set the \texttt{CLIENT\_APP\_URL} variable with the \URL of the app.
	\begin{lstlisting}
		CLIENT_APP_URL="<url>"\end{lstlisting}

\section{Sending for detection}
	\begin{figure}[h]
	\centering
	\includegraphics[width=15cm]{images/eida_send_manifest.png}
	\caption{Capture d'écran du formulaire d'envoi d'une numérisation dans l'application \eida}
	\label{fig:eida_send_manifest}
	\end{figure}

	\subsection{IIIF manifest}
	The user creates a \textbf{witness}, fills in the form and submit a digitization (either a PDF file, a \iiif manifest or images).

	The images are downloaded and the \textbf{\eida \iiif manifest} is generated: the request to the \api is not sent until all of the images are downloaded, to avoid errors and annotation of incomplete manifests.

	\subsection{annotate\_wit function}
	To send requests to \exapi from the \eida application, we created a function in \texttt{vhs-platform/vhsapp/utils/iiif/annotation.py}. The function sends a \textbf{POST request} to the \api endpoint \texttt{/run\_detect}, which launches detection on the images from the digitization.
	
	The request takes as arguments the \textbf{\URL of the manifest} we want annotated, and the \textbf{abbreviation} of the witness type.
	
	The \texttt{event} argument is used with the method \texttt{.wait()} to ensure the function \textbf{waits for an event to be set} before sending the request to the \api – this event will be set when the images from the previous step are all downloaded.
	
	\begin{lstlisting}[language=Python]
	def annotate_wit(event, witness_id, wit_abbr=MS_ABBR, version=MANIFEST_AUTO):
		wit_type = MS if wit_abbr == MS_ABBR else VOL
		api_endpoint = f"{GPU_URL}/run_detect"
	
		manifest_url = (
			f"{VHS_APP_URL}/{APP_NAME}/iiif/{version}/{wit_type}/{witness_id}/manifest.json"
		)
		data = {"manifest_url": manifest_url, "wit_abbr": wit_abbr}
	
		event.wait()
		requests.post(url=api_endpoint, data=data)
		
		return print(f"Witness {witness_id} sent for diagram extraction")\end{lstlisting}
	
	\subsection{Automating the annotation request}
	To launch annotation automatically after the user submits a digitization, we added a \textbf{background task} to the save method of \texttt{Manifests}, \texttt{Pdf} and \texttt{Image} objects.
	
	We use \textbf{threading} to avoid an interruption of the other tasks of the app while waiting for a response to the request.

		\subsubsection{User submitted a IIIF manifest}
		For manifests, we modified the save method of the \texttt{ManifestManuscript} and the \texttt{ManifestVolume} classes in \texttt{vhs/vhs-platform/vhsapp/models/digitization.py}. Ideally, we would have a unique save method for both types.
		
		An \texttt{event} object is created and can be passed through the various \textbf{threads}: the first thread \texttt{t} calls the function \texttt{extract\_images\_from\_iiif\_manifest} (in \texttt{vhs/vhs-platform/ vhsapp/utils/iiif/download.py}) to download images and \textbf{set the event} when the function is done running.

		\begin{lstlisting}[language=Python]
	def extract_images_from_iiif_manifest(manifest_url, witness_ref, event):
		"""
		Extract all images from an IIIF manifest
		"""
		downloader = IIIFDownloader(manifest_url, witness_ref)
		downloader.run()
		event.set()\end{lstlisting}

		\texttt{event.set()} returns a boolean value: event is \texttt{True} when the function has \textbf{run to its completion}. Because we used an \texttt{event.wait()} in the \texttt{annotate\_wit} function, it cannot run unless this value is \texttt{True}.
		
		When the image extraction is done and the event is set, the thread \texttt{t2} starts, therefore \textbf{sending the newly generated manifest to the \api for annotation}.

		\begin{lstlisting}[language=Python]
	class ManifestManuscript(Manifest):
		manuscript = models.ForeignKey(Manuscript, on_delete=models.CASCADE)
		
		def save(self, *args, **kwargs):
			# Call the parent save method to save the model
			super().save(*args, **kwargs)
			
			event = threading.Event()
		
			# Run the async extraction of images from an IIIF manifest in the background using threading
			t = threading.Thread(
				target=extract_images_from_iiif_manifest,
				args=(
					self.manifest,
					f"{MS_ABBR}{self.manuscript.id}",
					event,
				),
			)
			t.start()
			
			t2 = threading.Thread(
				target=annotate_wit,
				args=(
					event,
					f"{self.manuscript.id}",
					f"{MS_ABBR}",
				),
			)
			t2.start()\end{lstlisting}
	
		\subsubsection{User submitted a PDF file}
		For PDF files, the process is the same as for manifests.
		The \texttt{event} is set at the end of the \texttt{pdf\_to\_img function}, when all pages are converted to images, in \texttt{vhs/vhs-platform/ vhsapp/utils/functions.py}.

		\begin{lstlisting}[language=Python]
	def pdf_to_img(event, pdf_name):
		"""
		Convert the PDF file to JPEG images
		"""
		pdf_path = f"{BASE_DIR}/{MEDIA_PATH}/{pdf_name}"
		
		# e.g. pdf_name = "volumes/pdf/filename.pdf" => "filename"
		pdf_name = pdf_path.split("/")[-1].split(".")[0]
		pdf_info = pdfinfo_from_path(pdf_path, userpw=None, poppler_path=None)
		page_nb = pdf_info["Pages"]
		step = 2
		try:
			for img_nb in range(1, page_nb + 1, step):
				batch_pages = convert_from_path(
				pdf_path,
				dpi=300,
				first_page=img_nb,
				last_page=min(img_nb + step - 1, page_nb),
				)
				for page in batch_pages:
					save_img(page, f"{pdf_name}_{img_nb:04d}.jpg")
					img_nb += 1
			event.set()
		except Exception as e:
			log(f"[pdf_to_img] Failed to convert {pdf_name}.pdf to images:\n{e}")\end{lstlisting}

		In \texttt{vhs/vhs-platform/vhsapp/models/digitization.py}, the \texttt{save} method of the \texttt{Pdf} class was modified to start a second thread when the function has run.
		
		\begin{lstlisting}[language=Python]
	class Pdf(Digitization):
		class Meta:
			verbose_name = "PDF File"
			verbose_name_plural = "PDF Files"
			abstract = True  # TODO: make this class not abstract
		
		def __str__(self):
			return self.pdf.name
		
		def save(self, *args, **kwargs):
			# Call the parent save method to save the model
			super().save(*args, **kwargs)
			# Run the PDF to image async conversion task in the background using threading
			# t = threading.Thread(target=self.to_img())
	
			event = threading.Event()
			
			t = threading.Thread(target=pdf_to_img, args=(event, f"{self.pdf.name}",))
			t.start()
		
			t2 = threading.Thread(
				target=annotate_wit,
				args=(
					event,
					f"{self.volume.id if 'vol' in self.pdf.name else self.manuscript.id}",
					f"{VOL_ABBR if 'vol' in self.pdf.name else MS_ABBR}",
				),
			)
			t2.start()\end{lstlisting}

		\subsubsection{User submitted images}
		\begin{lstlisting}[language=Python]
	def save(self, *args, **kwargs):
		if self.image:
			img = Image.open(self.image)
			# Check if the image format is not JPEG
			if img.format != "JPEG":
				# Convert the image to JPEG format
				self.image = convert_to_jpeg(self.image)
		# Call the parent save method to save the model
		super().save(*args, **kwargs)
		
		event = threading.Event()
	
		t = threading.Thread(
			target=annotate_wit,
			args=(
				event,
				f"{self.volume.id if 'vol' in self.image.name else self.manuscript.id}",
				f"{VOL_ABBR if 'vol' in self.image.name else MS_ABBR}",
			),
		)
		t.start()\end{lstlisting}
		
\section{Receiving annotations}
For the annotations to be returned \textbf{from \exapi to the \eida application} after detection, we create an \textbf{endpoint} in \texttt{vhs/vhs-platform/vhs/urls.py}.

\begin{lstlisting}[language=Python]
	path(
		f"{APP_NAME}/<str:wit_type>/<int:wit_id>/annotate/",
		receive_anno,
		name="receive-annotations",
	),\end{lstlisting}

The endpoint calls the function \texttt{receive\_anno} in \texttt{vhs/vhs-platform/vhsapp/utils/ views.py} which receives a \textbf{file} from a \textbf{POST request from \exapi}. When the file is received, \textbf{if it is a text file}, its content is \textbf{written in a file in the annotation directory}. If the file already exists, the content is rewritten.

When the annotation file is saved, the annotations are indexed by the function \texttt{index\_anno}. 

\begin{lstlisting}[language=Python]
	def receive_anno(request, wit_id, wit_type):
		if request.method == "POST":
			annotation_file = request.FILES["annotation_file"]
			file_content = annotation_file.read()
		
			if is_text_file(file_content):
				anno_path = f"{BASE_DIR}/{MEDIA_PATH}/{MS_ANNO_PATH if wit_type == 'manuscript' else VOL_ANNO_PATH}"
				
				try:
					with open(f"{anno_path}/{wit_id}.txt", "w+b") as f:
						f.write(file_content)
		
				except Exception as e:
					log(f"[receive_anno] Failed to open received annotations for {wit_type} #{wit_id}: {e}")
		
				manifest_url = (f"{VHS_APP_URL}/{APP_NAME}/iiif/v2/{wit_type}/{wit_id}/manifest.json")
				try:
					index_anno(manifest_url, wit_type, wit_id)
				except Exception as e:
					log(f"[receive_anno] Failed to index annotations for {wit_type} #{wit_id}: {e}")
					
				return JsonResponse({"message": "Annotation received and indexed."})
			else:
				return JsonResponse({"message": "Invalid request. File is not a text file."}, status=400)
		else:
			return JsonResponse({"message": "Invalid request."}, status=400)\end{lstlisting}

\section{Indexing annotations}
In \texttt{vhs/vhs-platform/vhsapp/utils/iiif/annotation.py}, the function \texttt{index\_anno} launches the indexation of the annotations, to go \textbf{from a \texttt{.txt} file to an annotated manifest}. 
A \textbf{GET request} is sent to retrieve the \json content of the manifest, which is sent to the \textbf{annotation server} through a \textbf{POST request}. 

\begin{lstlisting}[language=Python]
	def index_anno(manifest_url, wit_type, wit_id):
		try:
			manifest = requests.get(manifest_url)
			manifest_content = manifest.json()
		except Exception as e:
			log(f"[index_anno]: Failed to load manifest for {wit_type} n°{wit_id}: {e}")
	
		requests.post(f"{SAS_APP_URL}/manifests", json=manifest_content)
	
		try:
			requests.get(f"{VHS_APP_URL}/{APP_NAME}/iiif/v2/{wit_type}/{wit_id}/populate/")
		except Exception as e:
			log(f"[index_anno]: Failed to index {wit_type} n°{wit_id}: {e}")\end{lstlisting}

		\clearemptydoublepage

\backmatter
    \printacronyms[title=Liste des acronymes,toctitle=Acronymes]
    \printglossary
    \listoffigures
    \clearemptydoublepage
    \tableofcontents
\end{document}