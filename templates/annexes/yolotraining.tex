\section{What is \yolov?}
	\begin{itemize}
		\item Built to detect objects in images
		\item Trained with a dataset of real images (ImageNet)
		\item Default model based on 80 object classes
	\end{itemize}

The goal of the training is to \textbf{adapt the model} to \textbf{our images and our classes}, for it to detect the objects we want it to.

\section{Training dataset}
	\begin{itemize}
		\item \textbf{Annotate the objects} of interest on the images to \textbf{create labels} : the objects are defined by the \textbf{position and dimensions of the rectangular box} surrounding them;
		\item The label file is a \textbf{.txt} file;
		\item In the label file, each object corresponds to \textbf{one row}:
			\begin{verbatim}
				class x_center y_center width height
			\end{verbatim}
		\item Organize labels and images in directories: \textbf{one directory for images and one for labels}, each containing a \textbf{training, validation and testing dataset};
		\item \textbf{Training is 80\% of the dataset}, validation is either 20\% or 10\% if a testing dataset (10\%) is needed;
		\item The training and validation datasets must be \textbf{different data};
		\item The image file and labels file must \textbf{have the same name}.
	\end{itemize}

\section{Config file}
The \textbf{dataset.yaml} is the \textbf{configuration file for the training}. It contains:
	\begin{itemize}
		\item The \textbf{dataset directory path};
		\item The relative paths to the training, validation and test directories;
		\item A \textbf{class names dictionary}: in our case, the only class is \textbf{0: illustration}.
	\end{itemize}


\begin{verbatim}
	path: ../dataset # dataset root dir
	train: images/train  # training images (relative to 'path')
	val: images/val  # validation images (relative to 'path')
	test:  # test images (optional)
	
	# Classes names:
	0: illustration
\end{verbatim}

\section{Notes}
	\begin{itemize}
		\item The training dataset is made of the \textbf{images the researchers annotate};
		\item Our model needs to be able to detect \textbf{each illustration individually}rather than extracting illustrated pages as a unique illustration;
		\item Our model needs to be able to differentiate tables from diagrams;
		\item The model is trained on both images of \textbf{manuscripts and printed books}: does it impact the quality of the extraction?
	\end{itemize}
