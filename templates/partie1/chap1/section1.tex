% Contexte et objectifs du projet

\subsection{Un projet de recherche en humanités numériques}
        \subsubsection{Cadre du projet}
Les sciences astronomiques se développent, à travers les siècles, à l’intersection de besoins et de questionnements divers auxquels elles tentent de répondre. Des besoins politiques aux besoins religieux, ces disciplines existent au cœur de contextes culturels, géographiques, historiques variés et interconnectés, et produisent, dans le cadre de leur pratique, des supports éclectiques qui témoignent de l'évolution des idées, et des conceptions du monde et de l'univers. Les manuscrits et imprimés produits dans le cadre des sciences astronomiques, pour répondre aux besoins de ceux qui les emploient, comportent des tables, des textes et des diagrammes d’une grande richesse, à l’image de la diversité de leurs contextes de production et de leurs utilisations. 

Le projet \eida\footcite{EdIterAnalyserDiagrammes}, ayant pour vocation d'analyser ces sources dans le cadre de l'histoire des sciences, s’intéresse plus spécifiquement aux diagrammes astronomiques, en tant que patrimoine visuel et supports pour la circulation des idées et théories astronomiques et mathématiques. L’étude de ces circulations implique l’exploitation d’un corpus large, représentatif des bornes chronologiques et géographiques du projet, qui s’étendent de l’Europe à l’Asie, du \viii au \xviii siècle. Le projet \eida a pour objectif d’étudier la diversité des diagrammes produits, du point de vue de l’utilisation mais aussi du support et de la diffusion. Les diagrammes sont étudiés d’une point de vue épistémique, par la définition de différentes typologies de diagrammes, l’analyse de leurs utilisations dans divers contextes et l’étude des sujets mathématiques et astronomiques qui les implique ; ainsi que d’un point de vue documentaire, avec un intérêt pour leurs contextes de production, le rapport texte-diagramme dans les manuscrits, et la circulation des représentations à une échelle locale ou globale. 

Le projet s’appuie sur des sources latines, byzantines, arabes, persanes, hébraïques, sanskrites et chinoises, qui partagent des éléments communs permettant ainsi de retracer les évolutions des sciences astronomiques, et d’étudier leur continuité, leur diversité, et leurs connections à travers l’histoire de l’astronomie.

        \subsubsection{Le choix de la vision artificielle}
Profondément ancré dans les méthodes des humanités numériques, le projet \eida a pour objectif le développement d’une approche reposant sur la vision artificielle, qui permettrait d’appliquer au corpus une série de traitements en prévision de leur analyse et de leur exploitation. Le développement d’outils basés sur la vision artificielle a pour but de limiter les étapes manuelles de fouille et d’annotation dans le corpus de manuscrits, en automatisant les étapes d’extraction des diagrammes dans les numérisations, et leur transformation en objets aisément manipulables pour leur analyse, comme pour l’édition. La projet \eida repose ainsi, dans ses principes fondateurs, sur une approche interdisciplinaire mêlant histoire des sciences et vision artificielle, avec pour objectif l’analyse de sources historiques dans le domaine des sciences astronomiques et le développement d’outils pour l’étude des diagrammes.

Dans une perspective de science ouverte, le projet \eida prévoit le développement d’une application Web dédiée à l’extraction, la visualisation, l’étude et l’édition de diagrammes ; une plateforme dédiée à la communauté de la recherche qui permettra une utilisation des données produites et des outils développés par \eida via une interface web. Cette plateforme permettra à un public de chercheurs, d’étudiants ou d’amateurs d’exploiter les algorithmes développés dans le cadre du programme pour traiter leurs propres sources ; à l'image de la plateforme développée dans le cadre du projet \dishas, dédié aux tables astronomiques. 

        \subsubsection{Annotations et données ouvertes}
Le projet \eida ayant pour objectif de produire un modèle de vision par ordinateur performant pour la détection de diagrammes dans des numérisations de manuscrits, un jeu de données d'entraînement -- ou vérité de terrain -- est produit à partir d'annotations réalisées par les chercheurs du programme. Ce jeu de données est composé d'un ensemble d'images et de leurs fichiers d'annotation où sont localisés manuellement les diagrammes, selon les résultats espérés de la détection automatique. Cette vérité de terrain se veut, pour des meilleurs résultats, la plus exhaustive possible quant au matériau source, et constitue ainsi un jeu de données diversifié en termes d'images et d'annotations, aussi bien du point de vue de la période que de la provenance géographique, ou simplement du style de représentation. 

\eida prévoit ainsi de mettre à disposition ce jeu de données, pour permettre son exploitation par d'autres projets qui souhaiteraient entraîner un modèle de vision artificielle pour la détection d'objet dans des images. Il est envisagé, pour le bon référencement de ce jeu de données, de les référencer dans un catalogue à l'image de l'initiative HTR-United\footnote{HTR-United ne propose pas de cataloguer des jeux de données pour la \cv, il est cependant envisagé de prendre contact avec les porteurs de l'initiative pour discuter de la possible intégration des données du projet \eida. \cite{chagueHTRUnited}} pour la reconnaissance automatique des écritures manuscrites. L'annotation étant une étape cruciale des projets en \ml, l'ouverture de ces données permettra ainsi à d'autres projets de bénéficier du travail effectué dans le cadre d'\eida.

    \subsection{Avant EiDA : le projet DISHAS}
        \subsubsection{Étudier les tables astronomiques}
Précédant le projet \eida, \dishas s’intéresse au développement des sciences astronomiques, à leur circulation, à leur transmission à travers les cultures et les époques, par le prisme des tables. De la même manière que les diagrammes sont un objet d’étude permettant de retracer des traditions scientifiques à travers les cultures eurasiennes, les tables sont un objet caractéristique des sciences astronomiques, dont le contenu numérique rend possible l’alignement et la comparaison entre différentes sources pour une étude de la circulation des idées entre les cultures et les pratiques scientifiques.

Le projet \dishas emploie les méthodologies des humanités numériques et les applique au traitement des sources en histoire de l’astronomie. Dans cette optique, une application dédiée à l’édition et à l’analyse des tables astronomiques a été mise en ligne\footcite{HomeDISHASProject} : cette application propose une contextualisation historique, chronologique et géographique des tables\footcite{HistoricalNavigationDISHAS}, ainsi qu’une série d’outils plus spécifiques, pour leur étude par un public divers, plus ou moins spécialisé. Parmi ces outils dédiés au traitement des tables, \acrfull{dti} est destiné à la saisie de tables numériques et \acrfull{dips} propose l’analyse mathématique de tables astronomiques anciennes ; mettant ainsi à disposition de la communauté de la recherche les résultats du projet, pour que les données comme les outils puissent être utilisés dans le cadre de recherches dans le domaine de l’histoire des sciences. 

        \subsubsection{Approche par la vision artificielle}
L’approche des données visuelles par l’\ia est un élément clé du projet \dishas, ayant donné lieu au développement de tableTranscriber\footcite{dotTableTranscriberAutomaticPipeline2022}, un outil dédié à la détection par un algorithme de vision artificielle\footnote{Basé sur \cite{monnierDocExtractor2023}.} de la structure des tables, afin d’en transcrire le contenu manuscrit à l'aide d'un algorithme de \acrshort{htr}. L’objectif de tableTranscriber est ainsi d’automatiser la chaîne de traitement des numérisations de tables astronomiques manuscrites, afin d’obtenir des données au format \acrshort{xml}, \acrshort{csv} ou \acrshort{html}, aisément manipulables pour une analyse des sources. 

Cette volonté collaborative dans le développement d’outils numériques est centrale au projet \eida qui, comme son prédécesseur \dishas, vise à produire une recherche communautaire, aux acteurs scientifiques divers, à l’intersection des humanités numériques et de l’histoire des sciences.

    \subsection{Acteurs du projet}
        \subsubsection{SYRTE-PSL-Observatoire de Paris, CNRS}
\eida est un projet interdisciplinaire mené en collaboration avec de nombreux acteurs et institutions, pour en faire un projet réellement collaboratif dans la réalisation de ses objectifs. Au sein de l'Observatoire de Paris, l'équipe est composée d'historiens et d'ingénieurs affiliés au \acrfull{syrte}, ayant pour la plupart collaboré au projet \dishas, et possédant donc une expérience de la recherche en histoire des sciences, et en humanités numériques. L'équipe compte également des chercheurs indépendants et collaborateurs scientifiques externes, qui assurent la diversité du corpus en contribuant à \eida par leur expertise dans des sources issues d'aires géographiques et chronologiques qui s'ajoutent à celles représentées dans l'équipe d'histoire des sciences de l'Observatoire de Paris.

Du point de vue des institutions, le groupe de recherche en vision artificielle \imagine basé à l'\ponts, co-coordonne le projet \eida, et assure notamment les recherches en vision artificielle et apprentissage machine nécessaires à la réalisation des objectifs du projet. Le projet \vhs, coordonné par l'\acrlong{iscd}, est également partenaire d'\eida pour le développement des algorithmes de détection et de la plateforme dédiée aux chercheurs.

        \subsubsection{Algorithmes de vision par ordinateur : le groupe de recherche IMAGINE}
L'équipe \imagine -- avec laquelle l'équipe d'histoire des sciences avait déjà collaboré pour le développement de tableTranscriber dans le projet \dishas -- apporte au projet \eida son expertise dans le domaine de l'apprentissage profond et de la vision artificielle appliqués aux humanités numériques. Les recherches menées entre 2018 et 2022 dans le cadre du projet \enherit\footcite{EnhancingHeritageImage}, dédié à l'application d'algorithmes de vision à l'histoire de l'art pour la reconnaissance de motifs dans les œuvres d'art, sont le socle des développements menés pour les projets \eida et \vhs. 

Dans le cadre de leur participation à \eida, l'objectif des chercheurs d'\imagine est le développement de nouvelles approches\footnote{Basées sur les progrès récents de l'analyse par synthèse. \cite{monnierUnsupervisedLayeredImage2021}} pour permettre la décomposition et la vectorisation de dessins techniques. Ces nouvelles approches par vision par ordinateur devraient permettre l'analyse et l'édition de diagrammes astronomiques sans intervention humaine, avec une méthode plus flexible dans son apprentissage, qui allègerait ainsi les étapes d'annotation nécessaires à l'apprentissage profond.

Ces recherches en apprentissage faiblement supervisé sont également menées dans le cadre du projet \vhs, auquel l'équipe \imagine contribue également, avec un accent mis sur la recherche de similarité dans les images et dans le texte, qui devraient permettre une navigation facilitée des corpus étudiés, et qui seront également exploitées dans \eida.


        \subsubsection{Le projet ANR VHS}
Le projet \vhs réunit des chercheurs en histoire des sciences et en vision artificielle pour l'étude de la circulation des savoirs scientifiques au Moyen-Âge et à la période moderne. Coordonné par Alexandre Guilbaud, \vhs a pour objectif de développer des méthodes d'analyse de corpus illustrés par apprentissage profond, afin d'étudier la circulation, l'évolution et la réutilisation des images scientifiques dans le cadre de la transmission des savoirs à travers les époques et les zones géographiques, et d'analyser le rapport des images au contenu textuel des ouvrages dans lesquels elles figurent. Les méthodes développées par le projet ont pour objectif de limiter la nécessité des annotations pour l'obtention de résultats dans la recherche de similarité, qui permettraient ainsi de faciliter la navigation de corpus scientifiques par des regroupements et rapprochements basés sur les images et sur le texte.

Ces recherches s'accompagnent du développement d'une application Web, qui mettra à disposition de la communauté scientifique les outils et données produites dans le cadre du projet \vhs, afin que ceux-ci puissent être exploités par d'autres équipes travaillant dans le domaine des études visuelles, ou souhaitant apporter une analyse sur des corpus illustrés par le prisme de la circulation des motifs et des idées. 

 Les membres de l'équipe \imagine sont sollicités lors des prises de décision relatives aux travaux menées par les chercheurs\footnote{On considère l'annotation des diagrammes dans les manuscrits comme un travail de recherche.} et participent aux discussions relatives au développement de la plateforme. De plus, pour un développement plus fluide des applications \vhs et \eida, l'ingénieur chargé de ce développement pour le projet \vhs mène régulièrement ses travaux à l'Observatoire, pour faciliter les échanges entre les deux équipes. Ces collaborations permettent une mutualisation des recherches et expertises des différents projets, ainsi que le développement commun d'outils réutilisables qui bénéficient des regards variés et complémentaires des différents acteurs.