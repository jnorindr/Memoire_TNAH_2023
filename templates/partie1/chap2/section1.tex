% L'image comme source

\subsection{Construire un corpus d’images : enjeux et méthodes}
        \subsubsection{La numérisation, enjeu patrimonial}
L'application à un corpus de sources historiques de méthodes numériques repose sur une première étape cruciale dans le traitement de ces sources : la numérisation des collections patrimoniales. De la bibliothèque au musée, la numérisation du patrimoine culturel est un enjeu crucial depuis le début des années 1990\footcite{baujardNumerisationPatrimoineCulturel2017}. Pensée initialement comme un outil de préservation, de valorisation et d'accessibilité aux collections\footcite{richardProgrammeNumerisationBibliotheque1993}, la mise à disposition des collections au format numérique ouvre également la voie à de nouvelles méthodologies de la recherche, notamment dans le champs disciplinaire des humanités numériques : dans le cadre d'un projet tel qu'\eida, qui prévoit des traitements automatiques des sources par des algorithmes de vision par ordinateur, la disponibilité des sources numérisées est un principe fondateur de la démarche de recherche. 

La numérisation des données culturelles fait désormais partie des pratiques courantes des institutions culturelles, et plus particulièrement des bibliothèques, représentant une part intégrante du travail de conservation et de diffusion du patrimoine\footcite{claerrModeEmploi2017}. Dès le milieu 1990, des initiatives telles que Gallica\footcite{Gallica}, bibliothèque numérique de la \bnf, voient le jour, et mettent à disposition du publique, au format numérique, les collections de la bibliothèque. Dans les années qui suivent, des initiatives telles qu'Europeana\footcite{Europeana}, permettent le déploiement de projets de plus large envergure\footcite{claerrModeEmploi2017}, portant toujours cette responsabilité de préservation, diffusion et valorisation des collections dans un contexte international. La notion de patrimoine culturel numérique prend tout son sens, et Internet devient un espace permettant à des utilisateurs aux profils variés d'exploiter ces ressources numériques, et d'accéder à un patrimoine vaste, au-delà des limites physiques de la consultation des documents. Cependant, malgré les initiatives de soutien à la numérisation portées par les collectivités, ou même par l'État\footcite{claerrModeEmploi2017}, il est nécessaire de souligner les disparités, aussi bien sur le plan national qu'international, de ces entreprises de numérisation, souvent coûteuses : ainsi, il n'existe pas de réelle égalité entre les institutions dans la numérisation et la mise en ligne des collections, un biais qu'il est nécessaire de prendre en compte, et particulièrement dans des projets de recherche appuyés sur des sources aux bornes géographiques internationales. 

Dans le cas des corpus étudiés par le projet \eida, on compte, pour des sources qui représentent plusieurs centaines de milliers de documents, une somme de plusieurs centaines de documents numérisés, mis à disposition par les institutions patrimoniales européennes, chinoises et indiennes : nous constatons ainsi que les documents numérisés ne représentent qu'un fragment de la réalité des sources existantes, cependant, ces centaines de documents numérisés sont à la fois suffisantes et suffisamment représentatives pour le bon déroulé du projet.

Les collections patrimoniales des musées présentent un tout autre enjeu en termes de mise en ligne : les œuvres conservées, allant bien au-delà du format livre, ont une diversité qui redéfinit entièrement la notion de numérisation\footnote{Par numérisation, dans le contexte muséal, nous entendons essentiellement la prise de photographies de haute qualité des œuvres d'arts.}. Tout comme les bibliothèques, les institutions muséales font face, avec le développement des pratiques numériques, à une transformation des pratiques de conservation pour inclure ces méthodes. Outre les aspects documentaires de la numérisation, le Web est un lieu d'exposition à part entière, et la numérisation des œuvres présentent ainsi un intérêt stratégique en termes de valorisation et de visibilité\footcite{baujardNumerisationPatrimoineCulturel2017}. Dans le monde muséal, néanmoins, l'accès libre aux images n'est pas aussi systématique qu'en bibliothèque\footnote{Pour comparaison, Gallica représente aujourd'hui près de 10 millions de documents, et la base de la \acrfull{rmn} compte 600 000 œuvres numérisées.} : il existe bel et bien des projets de numérisation de grande envergure, tels que Google Arts \& Culture\footcite{GoogleArtsCulture}, publié en 2011. La vocation de ce type de projets, cependant, semble plutôt être la création d'espaces d'exposition dématérialisés permettant de mettre en valeur les œuvres pour des spectateurs, plutôt que la véritable mise à disposition de ces images pour un usage par des utilisateurs. Nous soulignons, notamment, dans le cas de Google Arts \& Culture, l'impossibilité de copier les images avec un clic droit, et l'absence de bouton pour le téléchargement des numérisations sur un certain nombre de pages : les numérisations sont faites pour être vues, et non utilisées\footnote{Certaines œuvres du projet sont disponibles sur Wikimedia, mais ces images en accès libre ne représentent qu'un fragment des œuvres numérisées dans le cadre du projet. De plus, les musées français sont très peu représentés dans cette sélection. \cite{CategoryGoogleArt}}. 

Une étude menée entre 2019 et 2020 par le Musée National de Tokyo souligne, en parallèle, les disparités dans la qualité des numérisations, qui ne permet ainsi pas de considérer qu'un objet numérisé ou photographié l'est nécessairement dans une qualité exploitable\footcite{sakaiDigitizingDisparityMuseum2021}, notamment parce que les besoins en termes de qualité ne sont pas les mêmes pour des œuvres en 2D et des œuvres en 3D, qui souffrent particulièrement de la mauvaise qualité des numérisations. Ainsi, en prenant en compte ces critères, il est nécessaire de soulever la question de la réelle exploitabilité des images mises en ligne, et particulièrement dans le cadre de projets en vision artificielle qui reposent, dans leurs fondations, sur la disponibilité et la qualité des images du corpus choisi. Des initiatives récentes tentent cependant de pallier à ces lacunes des institutions patrimoniales quant à la mise à dispostion des images, comme la création en 2015 d'une \api donnant accès aux reproductions de la base photographie de la \acrshort{rmn}\footcite{APIRMNGrandPalais}, mais la norme, pour ces images d'œuvres d'art, reste loin de l'\textit{open access}\footcite{mancaNouveauxDefisAgences2018}. 

        \subsubsection{Droit d'auteur et coût des images}
La numérisation et la mise en ligne de documents patrimoniaux implique également de prendre en compte la question des droits d'auteur, qui régit souvent les possiblités des institutions en termes de libre accès, et impacte par conséquent les projets, aussi bien du point de vue des exploitations possibles que de la publication de leurs résultats\footcite{jacquotDecrireTranscrireDiffuser2017}. 

Une œuvres originale est protégée au titre du droit d'auteur : ce droit comprend le droit moral, perpétuel et inaliénable, qui donne droit au respect de l'œuvre et du nom de son auteur, ainsi que le droit patrimonial, qui donne à l'auteur et ses ayants droit le droit d'autoriser ou non la reproduction et la diffusion de l'œuvre\footcite{sepetjanRespecterDroitPropriete2017}. Les droits patrimoniaux s'éteignent soixante-dix ans après la mort de l'auteur, et l'œuvre entre alors dans le domaine public\footnote{Il existe des cas particuliers, notamment pour les œuvres collectives ou les journaux, ainsi que pour les publications posthumes. \cite{GuidePratiquePour}.}. Il n'existe pas d'exception au droit d'auteur pour une numérisation mise en ligne sur Internet\footcite{sepetjanRespecterDroitPropriete2017}. Ces droits, qui s'appliquent à l'œuvre originale, ne sont pas nécessairement ceux qui nous intéressent dans le cadre d'un projet de recherche en histoire, dont les œuvres du corpus ont souvent rejoint le domaine public. La photographie d'une œuvre, cependant, peut faire l'objet du droit d'auteur, si celle-ci n'est pas considérée comme une reproduction servile.

Outre le droit d'auteur et les droits du photographe, les institutions peuvent exiger une redevance de réutilisation ou de prestation d'un service pour l'utilisation de leurs images, qui sert à couvrir, par exemple, les frais engagés pour la production du cliché, le traitement de la demande, ou la recherche iconographique\footcite{GuidePratiquePour}. Beaucoup d'institution, telles que la \bnf et l'\inha\footcite{denoyelleProposCoutImages2021}, proposent des tarifications particulières pour des projets académiques, scientifiques ou non commerciaux, et n'appliquent un coût qu'en cas de demande de numérisation d'œuvres qui ne sont pas déjà numérisées. Les projets de recherche peuvent ainsi faire le choix de construire leurs corpus en prenant en compte les documents déjà disponibles, et mis en ligne sous licence libre\footnote{Pour permettre aux chercheurs de traiter leurs numérisations sous droits avec les outils de la plateforme \eida, il a été décidé de proposer une option -- sous la forme d'un champs à cocher dans le formulaire de saisie -- pour que la numérisation envoyée ne soit pas rendue publique.}.

    \subsection{Images et données ouvertes}
        \subsubsection{Licences Creative Commons}
Pour favoriser la libre circulation des images sur le Web, des licences ont été créées pour encadrer la réutilisation des numérisations. Dès 2001, les licences Creative Commons\footnote{Ces licences ne sont plus reconnues par l'État français depuis 2016. \cite{denoyelleSavoirLicenceEtalab2021}.} sont créées avec la vocation de favoriser la diffusion et le parage des images dans un contexte numérique, en assouplissant notamment les droits de propriété intellectuelle pour s'adapter à ce nouveau cadre. Ces licences se basent sur un système de quatre options qu'il est possible de combiner pour créer six licences différentes. Ces options sont attribution (BY), non commercial (NC), sans modification (ND) et partage à l’identique (SD)\footcite{GuidePratiquePour}. 

À partir de 2011, l'État français créée ses propres licences ouvertes, comme la licence unique Etalab, qui vise à \og permettre et encourager la réutilisation des données publiques\footcite{denoyelleSavoirLicenceEtalab2021} \fg. Une deuxième version, publiée en 2017, se veut compatible avec la licence CC BY, c'est-à-dire une licence ouverte avec attribution de l'œuvre à son auteur.

Les données ouvertes sont un enjeu central des politiques culturelles : dans un contexte patrimonial, ces licences libres, conçues pour adapter les droits de propriété intellectuelle au contexte particulier du numérique, permettent ainsi la circulation, la diffusion et le partage de l'information. Il n'existe pas encore de cadre juridique international régissant l'utilisation des images produites par numérisation de collections patrimoniales, ainsi, les institutions sont soumises au droit spécifique du pays qui les abrite. En l'absence d'un encadrement global des pratiques, les utilisateurs -- et notamment la communauté de la recherche -- font face aux exceptions propres à chaque contexte national, qui représentent encore un obstacle. Une réflexion autour de ce cadre juridique international permettrait, en parallèle, d'harmosniser à une échelle globale les formats des données ouvertes\footcite{benhamouDroitAuteurMusees2016}, pour en faciliter la réutilisation par les institutions et par les projets.

        \subsubsection{Enjeux techniques : images et métadonnées}
Le document numérique obtenu à la suite d'une processus de numérisation est, en réalité, constitué de plusieurs fichiers répondant à des besoins différents pour la mise en ligne. Chacun de ces éléments représentent leurs propres enjeux techniques, dont la gestion impacte la qualité et l'exploitabilité des images mises en ligne à la suite d'entreprises de numérisation.

Les fichiers image issus d'un processus de numérisation peuvent avoir plusieurs formats\footnote{Un format décrit la manière dont les informations sont organisées dans un fichier.} aux propriétés similaires mais différentes. En premier lieu, il est de bonne pratique de privilégier un format ouvert\footcite{besseNumerisationMasseVers2019}, par opposition à un format propriétaire qui nécessite un logiciel spécifique pour être lu. Le choix d'un format ouvert permet d'assurer l'accessibilité des données produites. Les formats .tiff et .jpeg sont favorisés dans le contexte de numérisation des collections patrimoniales, parce qu'ils permettent notamment la compression des fichiers\footnote{La compression sans perte s'effectue par identification et suppression des redondances, c'est-à-dire des pixels identiques. Cette méthode n'est pas irréversible, il est possible de décompresser le fichier.} tout en maintenant la haute qualité des images\footcite{DigitalImages2020}. Les choix effectués par les institutions impactent la qualité des numérisations produites qui, sur un plus long terme, impacte aussi bien la préservation des données, l'expérience des utilisateurs qui consultent les images, et l'exploitabilité des fichiers par des projets. La question de la qualité des images est cruciale dans le cadre d'un projet de recherche en vision artificielle, puisque celle-ci a la possiblité d'impacter les performances de l'algorithme utilisé, et d'affecter négativement les interprétations du modèle\footcite{bergstromImageQualityComputer2023}.

Les fichiers images sont systématiquement accompagnés de métadonnées de diverses catégories, qui visent à décrire et identifier le document. Bien qu'il n'existe pas de standard international obligatoire concernant les métadonnées de numérisation, la standardisation des fichiers de métadonnées selon certains standards spécifiques permet de fluidifier la communication et les échanges entre institutions, dans une perspective interopérable qui permettrait de sortir des silos de données. Les métadonnées sont de plusieurs types : descriptives\footnote{Pour identifier le contenu du document numérique, le rattacher au document original.}, administratives\footnote{Pour gérer les droits d'accès, préserver les informations techniques, suivre les modifications du fichier et garantir son intégrité.}, structurelles\footnote{Pour rattacher les fichiers entre eux et reconstituer la structure du document.} ; elles ont pour vocation de gérer tous ces aspects du document numérique. Ces métadonnées sont cruciales pour l'identification et la classification des documents numérisés, mais aussi pour permettre la recherche et la navigation à travers les collections. Malgré leur importance, les différentes institutions ont une gestion profondément hétérogène de leurs métadonnées, sans uniformisation nationale ou internationale, et avec une disparité profonde entre les différents types d'institutions patrimoniales.

Pour pallier à cette disparité dans la gestion des métadonnées et au manque de standards pour le partage des images en ligne, des initiatives internationales ont vu le jour, visant à mettre fin à cette gestion en silo des données des institutions patrimoniales pour faciliter les échanges et la mise à disposition des données dans une optique de libre accès.
