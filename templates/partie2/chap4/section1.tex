% Réseaux de neurones et computer vision

\subsection{Des \og neurones \fg pour le \dl}
Les modèles de vision artificielle pour des tâches telles que la détection d'objet, la recherche de similarités ou la détection de lignes reposent souvent, en leur cœur, sur des réseaux de neurones\footnote{Un neurone, dans un contexte de \ml, désigne une unité de calcul indépendante mise en série -- ou en réseau -- avec d'autres unités de calcul, avec lesquelles elle communique par vagues pour effectuer des calculs tout en se corrigeant mutuellement. Le termes de \og neurone \fg est employé par assimilation avec le fonctionnement du cerveau, bien que les modèles actuels ne s'en inspirent plus dans leur construction. \cite{azencottIntroductionAuMachine2018}} plus ou moins profonds. Un réseau de neurones artificiels est un modèle paramétrique plus ou moins complexe\footcite{azencottIntroductionAuMachine2018} composé d'une couche d'entrée recevant les données brutes, puis d'une ou de plusieurs couches de calcul traitant les données en se corrigeant mutuellement, et d'une couche de sortie proposant une prédiction à partir des données d'entrée et des calculs effectués. Un réseau de neurones est dit \og profond \fg lorsqu'il compte de multiples couches\footcite{azencottIntroductionAuMachine2018}. La prédiction se fait à l'aide d'un système de poids, qui correspondent aux paramètres d'une couche pour la transformation qu'elle applique aux données d'entrée\footcite{cholletApprentissageProfondAvec2020a}, et qui a pour objectif de calculer la prédiction la plus juste, c'est-à-dire d'associer l'entrée avec une cible\footnote{Pour un algorithme de détection d'images dans les manuscrits, l'entrée est l'image de la page, et la cible est l'étiquette \og Illustration \fg qui sera attribuée à la zone de l'image où une illustration est détectée.} par une série de transformations. L'algorithme corrige ses poids en cas de prédiction fausse : il s'agit de l'apprentissage, qui signifie donc la recherche d'un ensemble de valeurs pour les poids de toutes les couches du réseau de sorte que les résultats obtenus soient satisfaisants.
    
\subsection{Réseaux de neurones à convolution}
Les modèles mis en avant dans ce mémoire de recherche reposent tous plus précisément sur des réseaux de neurones à convolution, ou \cnn, développés pour reconnaître des motifs visuels dans des images avec un minimum de traitements appliqués. Ces derniers sont favorisés pour les tâches de vision par ordinateur, et pour les tâches de perception en général\footcite{cholletApprentissageProfondAvec2020a} Ces réseaux sont composés de deux types de neurones agencés en plusieurs couches : les neurones de traitement, dédiés à traiter chacun une portion de l'image, et les neurones de mise en commun des sorties, dits de \textit{pooling}\footcite{goodfellowDeepLearning2016}. Les couches de convolution\footnote{Une couche de convolution est une couche constituée de copies d'un même neurone qui ne prend en compte qu'une partie le l'entrée.} pour la vision artificielle ont pour spécificité d'être basées sur des fragments qui ne représentent que quelques pixels des images d'origine\footnote{Ces motifs locaux peuvent être les bords, les textures, et d'autres éléments. \cite{cholletApprentissageProfondAvec2020a}} : elles apprennent des motifs locaux, à l'inverse des couches entièrement connectées qui apprennent des motifs globaux. Les \cnn ont besoin de moins d'exemples d'apprentissage, parce qu'ils ont la capacité d'apprendre un motif et de le reconnaître quelle que soit sa position. Les \cnn requièrent donc un volume de mémoire moins important, pour une efficacité supérieure\footcite{goodfellowDeepLearning2016}. 
    
Les \cnn ont la capacité d'apprendre des motifs locaux, puis d'apprendre dans les couches suivantes des motifs plus grands qui en découlent, apprenant ainsi des concepts visuels de plus en plus abstraits, et de plus en plus complexes\footcite{cholletApprentissageProfondAvec2020a} : l'intérêt dans le cadre d'un apprentissage pour la vision artificielle réside donc dans cette possibilité d'apprendre des motifs invariants par translation\footnote{\og  L'invariance par translations est une propriété fondamentale partagée par la quasi-totalité des opérations de traitement d'images. Elle exprime le fait qu'une information visuelle sera traitée de la même façon, quelle que soit sa localisation dans l'espace. En effet, si on bouge le capteur, chaque objet sera déplacé dans l'image, mais devra être traité de la même façon que précédemment. \fg \cite{ronseInvarianceParTranslations}} et spatialement hiérarchiques, deux caractéristiques fondamentales du monde visuel\footcite{cholletApprentissageProfondAvec2020a}.

