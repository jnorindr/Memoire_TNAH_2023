\chapterNo{Résumé}
\addcontentsline{toc}{chapter}{Résumé}
\medskip	

Ce mémoire a été réalisé dans le cadre du Master Technologies numériques appliquées à l'histoire de l'École nationale des chartes. Il est rédigé suite à un stage de quatre mois à  l'Observatoire de Paris au sein du projet \acrshort{eida}, portant sur les diagrammes astronomiques de tradition ptoléméenne du \viii au \xviii siècle et intégrant des techniques de vision artificielle pour le traitement de ses sources. Ce mémoire expose le développement d'outils pour cette intégration, et la conception d'une chaîne de traitement accompagnée de normes et de méthodes pour l'application d'algorithmes de vision dans le cadre d'un projet de recherche en humanités.\\

\textbf{Mots-clés~:} histoire des sciences~; diagrammes astronomiques~; images~; \textit{machine learning}~; vision artificielle~; apprentissage profond~; YOLOv5~; réseau de neurones artificiels~; automatisation~; Python~; programmation modulaire~; IIIF~; API~; Flask.\\

\textbf{Informations bibliographiques~:} Jade Norindr, \textit{Le traitement des sources historiques par la vision artificielle. L'exemple des manuscrits d'astronomie de tradition ptoléméenne}, mémoire de master \og Technologies numériques appliquées à l'histoire~\fg, dir. Maxime Challon, École nationale des chartes, 2023.	
\clearemptydoublepage